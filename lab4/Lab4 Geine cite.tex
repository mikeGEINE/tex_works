% !TEX program = xelatex

\documentclass[a4paper, 14pt]{extarticle}
\usepackage[english, russian]{babel}
\usepackage{fontspec} %% подготавливает загрузку шрифтов Open Type, True Type и др.
\defaultfontfeatures{Ligatures={TeX},Renderer=Basic} %% свойства шрифтов по умолчанию
\setmainfont[Ligatures={TeX,Historic}]{Times New Roman} %% зада¨eт основной шрифт документа
\setsansfont{Comic Sans MS} %% зада¨eт шрифт без засечек
\setmonofont{Courier New}
\usepackage{indentfirst} % отменяет отсутствие абзацного отступа для первого абзаца.
\frenchspacing % Правильные отступы для кавычек, пробелов, знаков препинания.
\usepackage{bmstu-lab}
\hypersetup{
    linkcolor=black
}
\usepackage{cite}

\begin{document}
    \graphicspath{{images/}{images2/}} % папки с картинками

    \worknumber{3.1}
    \variant{4}
    \workname{Библиография в \LaTeX}
    \discipline{Автоматизация процессов разработки научно-технической документации}
    \group{ИУ6-64Б}
    \author{М.А.Гейне}
    \tutor[Преподаватель]{Т.А.Ким}
    \bmstutitlelab

    \section{Библиография с использованием Cite}
    Ссылаюсь на работу источника \cite{Sotnikov}
    
    \begin{thebibliography}{1}
        \bibitem{Sotnikov} 
        Сотников А.А. Cпособ повышения эффективности вычислительных комплексов
        цифрового имитационного моделирования гидроакустической обстановки в
        реальном масштабе времени. // Наука и образование. МГТУ им. Н.Э. Баумана.
        Электронный журнал 2013. № 2. DOI: http://dx.doi.org/10.7463/0213.0531784.
        
    \end{thebibliography}

\end{document}