% !TEX program = xelatex

\documentclass[a4paper, 14pt]{extarticle}
\usepackage[english, russian]{babel}
\usepackage{fontspec} %% подготавливает загрузку шрифтов Open Type, True Type и др.
\defaultfontfeatures{Ligatures={TeX},Renderer=Basic} %% свойства шрифтов по умолчанию
\setmainfont[Ligatures={TeX,Historic}]{Times New Roman} %% зада¨eт основной шрифт документа
\setsansfont{Comic Sans MS} %% зада¨eт шрифт без засечек
\setmonofont{Courier New}
\usepackage{indentfirst} % отменяет отсутствие абзацного отступа для первого абзаца.
\frenchspacing % Правильные отступы для кавычек, пробелов, знаков препинания.
\usepackage{bmstu-lab}
\hypersetup{
    linkcolor=black
}
\usepackage[
backend=biber,
bibencoding=utf8,
sorting=ynt,
maxcitenames=2,
bibstyle=gost-authoryear,
citestyle=gost-authoryear
]{biblatex}
\addbibresource{bib1.bib}
\usepackage{acronym}

\renewcommand*{\aclabelfont}[1]{\textbf{\acsfont{#1~---}}}


\begin{document}
    \graphicspath{{images/}{images2/}} % папки с картинками

    \worknumber{3.2}
    \variant{4}
    \workname{библиография в \LaTeX}
    \discipline{Автоматизация процессов разработки научно-технической документации}
    \group{ИУ6-64Б}
    \author{М.А.Гейне}
    \tutor[Преподаватель]{Т.А.Ким}
    \bmstutitlelab

    \section{Библиография с использованием Biblatex}
    Это сегмент текста, в котором я собираюсь процитировать тексты. Вот здесь я ссылаюсь на книгу \cite{harper2006inside}.
    А тут я хочу привести в пример обзорную статью об умном доме: \cite{suresh2015review}.

    Эта статья для тех, кто особенно заинтересован в технологии: \cite{sripan2012research}.
    \nocite{robles2010applications}
    
    \printbibliography

    \section{Создание списка сокращений}

    \begin{acronym}
        \acro{mp}[МП]{микропроцессор}
        \acrodefplural{mp}[МП]{микропроцессором}
        
        \acro{alu}[АЛУ]{арифметико-логическое устройство}
        \acrodefplural{alu}[АЛУ]{арифметико-логическим устройством}

        \acro{mps}[МПС]{микропроцессорная система}
        \acrodefplural{mps}[МПС]{микропроцессорной системой}

        \acro{mk}[МК]{микроконтроллер}
        \acrodefplural{mk}[МК]{микроконтроллером}

        \acro{evm}[ЭВМ]{электронно-вычислительная машина}
        \acrodefplural{evm}[ЭВМ]{электронно-вычислительной машины}
    \end{acronym}

    \Acl{mp} –-- устройство, осуществляющее программно-управляемую
     обработку данных и выполняемое на интегральной технологии на одном кристалле. 
     В состав входит \ac{alu}, регистры, генератор тактовых импульсов, шинный интерфейс, 
     устройство управления.
    
    Комплекс аппаратных средств, в который входит микропроцессор, устройствово ввода, 
    устройство постоянной и оперативной памяти, устройство вывода, каналы обмена данными, 
    средства взаимодействия с оператором называется~\acp{mps}.

    \Acl{mk} –-- однокристальная \acs{evm}, которая характ. набором  компонентов, выполняемых
     в едином технологическом процессе. В состав входит: устройство ввода, устройство постоянной
     и оперативной памяти, устройство вывода, генератор тактовых импульсов, блок регистров, периферийные устройства.

    \tableofcontents
\end{document}