\documentclass[a4paper, 12pt]{article}
\usepackage{cmap}
\usepackage{mathtext}
\usepackage{amsmath,amssymb}
\usepackage[T2A]{fontenc}
\usepackage[utf8]{inputenc}
\usepackage[english, russian]{babel}
\usepackage{bmstu-lab}


\begin{document}
\graphicspath{{images/}{images2/}} % папки с картинками

\worknumber{1}
\variant{4}
\workname{Математика в \LaTeX}
\discipline{Автоматизация процессов разработки научно-технической документации}
\group{ИУ6-64Б}
\author{М.А.Гейне}
\tutor[Преподаватель]{Т.А.Ким}
\bmstutitlelab

\hypersetup{
    linkcolor=black
}

\section{Матрицы}
\[ \begin{bmatrix}
    \infty & 13 \\
    12 & \frac{1}{\infty} \\
    13 & \infty
\end{bmatrix} \]

\section{Интеграл}
\[ \int^{\delta_{\alpha0}}_{-\delta_{\alpha0}}f_{\alpha}(\delta_\alpha)d\delta_\alpha \geqslant P_\alpha,\]
где $P_D$, $P_\alpha$, $P_\gamma$, $P_\upsilon $ - заданные пороговые

\section{Сумма}

\[ \frac{\sum^N_{i=0}x_i + 77}{1 + \Delta N} \]
\[ \frac{\displaystyle\sum^N_{i=0}x_i + 77}{1 + \Delta N} \]

\section{Два столбика}

\begin{align}
    &\frac{\displaystyle\sum^N_{i=0}x_i + 77}{1 + \Delta N} & 
    \int^{\delta_{\alpha0}}_{-\delta_{\alpha0}}f_{\alpha}(\delta_\alpha)d\delta_\alpha &\geqslant P_\alpha \\
    &b_1 < b_2,~ b_2 \text{-- простое}_\text{число} &
    b_3&\& b_4
\end{align}


\section{Почти то же самое}

\begin{equation}\label{eq:ref}
    \begin{aligned}
        &\frac{\sum^N_{i=0}x_i + 77}{1 + \Delta N} & 
        \int^{\delta_{\alpha0}}_{-\delta_{\alpha0}}f_{\alpha}(\delta_\alpha)d\delta_\alpha &\geqslant P_\alpha \\
        &b_1 < b_2,~ b_2 \text{-- простое}_\text{число} &
        b_3&\&b_4
    \end{aligned}
\end{equation}

На странице \pageref{eq:ref} есть формулы \eqref{eq:ref}.


\end{document}
