% !TEX program = xelatex

\documentclass[a4paper, 14pt]{article}
\usepackage{bmstu-lab}
\hypersetup{
    linkcolor=black
}
% \usepackage{showframe}


\begin{document}
    \graphicspath{{images/}{images2/}} % папки с картинками
    \renewcommand{\figurename}{Рисунок}

    \worknumber{1}
    \variant{67}
    \workname{Проектирование устройств управления с жесткой логикой}
    \discipline{Основы проектирования устройств ЭВМ}
    \group{ИУ6-64Б}
    \author{М.А.Гейне}
    \tutor[Преподаватель]{А.Ю.Попов}
    \bmstutitlehome

    \textbf{Цель работы:} изучить методы проектирования устройств управления с жесткой логикой.
    \section*{Задание}
    В ходе выполнения домашнего задания необходимо разработать устройство
    управления схемного типа, обрабатывающий входное командное слово
    $ C ={ABCDEF}$ и выдающий сигналы управления $M={M_0,…,M_{k-1}}$
    операционному блоку в соответствии с приведенной в индивидуальном задании
    логикой работы.
    Домашнее задание выполняется в несколько этапов.

    \renewcommand{\labelenumii}{\Alph{enumii}}
    \begin{enumerate}
        \item \begin{enumerate}
            \item По диаграмме переходов автомата (Приложение 1) и описанию условий
            переходов и активных сигналов (дополнительный файл варианты.pdf),
            определить тип управляющего автомата (автомат Мили или Мура, смешанный).
            Выбор обосновать.
            \item Произвести кодирование состояний управляющего автомата. Составить схему
            переходов/состояний полученного автомата. Схему представить в отчете. 
        \end{enumerate} 
        \item Разработать описание устройства управления на языке VHDL, для чего
        использовать приведенные в Приложении 2 шаблоны для автоматов Мили и
        Мура.
        Разработать тестовое описание для устройства, представляющее собой
        генератор входных сигналов (см. Приложение 3). Тестовое описание должно
        обеспечивать проверку всех ветвей автомата
        \item \begin{enumerate}
            \item Установить ПО ModelSim PE (или аналогичный продукт: Xilinx ISE, Altera
            Quartus)
            \item Выполнить моделирование полученного теста в ПО ModelSim PE. Результаты
            моделирования представить в отчете.
        \end{enumerate}
    \end{enumerate}

    \section*{Вариант 67}
    \begin{table}[H]
        \caption{Варианты диаграмм и активных сигналов}
        \begin{tabular}{|c|c|c|l|l|l|l|l|}
            \hline
            \multirow{2}{*}{Вариант} & \multirow{2}{*}{Диаграмма переходов} & \multicolumn{6}{c|}{Активные сигналы M в состоянии}                                                                                                      \\ \cline{3-8} 
                                     &                                      & S1                     & \multicolumn{1}{c|}{S2} & \multicolumn{1}{c|}{S3} & \multicolumn{1}{c|}{S4} & \multicolumn{1}{c|}{S5} & \multicolumn{1}{c|}{S6} \\ \hline
            \multicolumn{1}{|l|}{67} & \multicolumn{1}{l|}{3}               & \multicolumn{1}{l|}{2} & 0                       & 1,7                     & 5,6                     & 3                       & 4                       \\ \hline
        \end{tabular}
    \end{table}

    \begin{table}[H]
        \caption{Условия переходов и наименование отладочной платы}  
        \begin{tabular}{|l|l|l|l|l|l|l|}
            \hline
            \multicolumn{1}{|c|}{\multirow{2}{*}{\textbf{Вариант}}} & \multicolumn{1}{c|}{\multirow{2}{*}{\textbf{Название отладочной платы}}} & \multicolumn{5}{c|}{\textbf{Активные сигналы C в состоянии}}                                                                                                                 \\ \cline{3-7} 
            \multicolumn{1}{|c|}{}                                  & \multicolumn{1}{c|}{}                                                    & \multicolumn{1}{c|}{\textbf{У1}} & \multicolumn{1}{c|}{\textbf{У2}} & \multicolumn{1}{c|}{\textbf{У3}} & \multicolumn{1}{c|}{\textbf{У4}} & \multicolumn{1}{c|}{\textbf{У5}} \\ \hline
            \multirow{5}{*}{67}                                     & \multirow{5}{*}{Spartan3}                                                & @                                & E\_F                             & CD                               & \_A\_C                           & @                                \\ \cline{3-7} 
                                                                    &                                                                          & \multicolumn{1}{c|}{\textbf{У6}} & \textbf{У7}                      & \textbf{У8}                      & \textbf{У9}                      & \textbf{У10}                     \\ \cline{3-7} 
                                                                    &                                                                          & D\_B                             & @                                & A\_C                              & C\_D                            & @                                \\ \cline{3-7} 
                                                                    &                                                                          & \textbf{У11}                     & \textbf{У12}                     & \textbf{У13}                     & \textbf{У14}                     & \textbf{У15}                     \\ \cline{3-7} 
                                                                    &                                                                          & A+C                              & \_B                               & @                                & DF                            & @                              \\ \hline
            \end{tabular}  
    \end{table}
    
    \begin{table}[H]
        \caption{Активные сигналы для переходов}
        \begin{tabular}{|l|l|l|l|l|l|}
        \hline
        \multicolumn{1}{|c|}{\multirow{2}{*}{\textbf{Вариант}}} & \multicolumn{5}{c|}{\textbf{Активные сигналы M в состоянии}}                                                                                                                 \\ \cline{2-6} 
        \multicolumn{1}{|c|}{}                                  & \multicolumn{1}{c|}{\textbf{У1}} & \multicolumn{1}{c|}{\textbf{У2}} & \multicolumn{1}{c|}{\textbf{У3}} & \multicolumn{1}{c|}{\textbf{У4}} & \multicolumn{1}{c|}{\textbf{У5}} \\ \hline
        \multirow{5}{*}{67}                                     &                                  &                                  &                                  &                                  &                                  \\ \cline{2-6} 
                                                                & \multicolumn{1}{c|}{\textbf{У6}} & \textbf{У7}                      & \textbf{У8}                      & \textbf{У9}                      & \textbf{У10}                     \\ \cline{2-6} 
                                                                &                                  &                                  &                                  & 5, 6                             & 5, 7                             \\ \cline{2-6} 
                                                                & \textbf{У11}                     & \textbf{У12}                     & \textbf{У13}                     & \textbf{У14}                     & \textbf{У15}                     \\ \cline{2-6} 
                                                                &                                  & 4                                &                                  &                                  &                                  \\ \hline
        \end{tabular}
    \end{table}

    \imageinsert{general transitions scheme}{0.7}{Общая диаграмма переходов}{fig:gen_trans}
    \pagebreak

    \section{Определение заданного автомата}
    В соответствии с вариантом домашнего задания была составлена диаграмма переходов состояний автомата, приведённая на рисунке \ref{fig:trans_diag}.
    \imageinsert{transitions diagram}{0.52}{Диаграмма переходов состояний заданного автомата}{fig:trans_diag}
    Так как выходные сигналы определены в некоторых случаях только состояниями (как в состоянии S2), а в некоторых 
    случаях определены ещё и входными сигналами (переход У9), то можно сказать, что заданный автомат является автоматом смешанного типа.

    \section{Разработка устройства}
    После того, как был определён тип автомата, на языке VHDL был описан заданный автомат.
    Автомат имеет асинхронные входы и выходы. Код устройства приведён ниже.
    \codeinsert{'codes/state_machine.vhd'}

    \section{Тестирование автомата}
    Для проверки корректности работы автомата был создан testbench, который проверяет все условия перехода устройства. Код устройства приведён ниже.
    \codeinsert{'codes/state_machine_test.vhd}

    Данное устройство было просимулировано, временная диаграмма приведена на рисунке \ref{fig:time_diag}.
    \imageinsert{time_diagram}{0.62}{Временная диаграмма теста}{fig:time_diag}
    Проанализировав временную диаграмму можно убедиться, что устройство работает корректно. Можно заметить, что устройство меняет своё состояние только с приходом положительного фронта.
    При этом входы и выходы устройства не зависят от синхросигналов. По этой причине можно увидеть, что в одном такте на выходах устройства может быть несколько разных наборов данных. 
    К примеру, такая ситуация наблюдается на интервале 135ns-145ns, когда фронтов синхросигнала нет, но при этом выходные сигналы меняются вследствие изменения входных сигналов в это время.

    \section*{Выводы}
    Изучены принципы проектирования устройств управлнения с жесткой логикой.

    Изучены способы задания автоматов Мили и Мура на языке VHDL.

    Изучены устройства с синхронными и асинхронными входами и выходами.

    Разработан автомат смешанного типа. Задано его текстовое описание на языке VHDL.

    Проведено тестирование разработанного устройства с помощью VHDL Testbench. 

    Проведено моделирование тестирующего устройства в среде ISim. По результатам моделирования было установлено, что разработанное устройство работает корректно.
\end{document}