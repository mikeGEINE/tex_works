% !TEX program = xelatex

\documentclass[a4paper, 14pt]{extarticle}
\usepackage[english, russian]{babel}
\usepackage{fontspec} %% подготавливает загрузку шрифтов Open Type, True Type и др.
\defaultfontfeatures{Ligatures={TeX},Renderer=Basic} %% свойства шрифтов по умолчанию
\setmainfont[Ligatures={TeX,Historic}]{Times New Roman} %% зада¨eт основной шрифт документа
\setsansfont{Comic Sans MS} %% зада¨eт шрифт без засечек
\setmonofont{Courier New}
\usepackage{indentfirst} % отменяет отсутствие абзацного отступа для первого абзаца.
\frenchspacing % Правильные отступы для кавычек, пробелов, знаков препинания.
\usepackage{bmstu-lab}
\usepackage{fancyvrb}
\usepackage{fvextra}
\usepackage{verbatim}
\hypersetup{
    linkcolor=black
}


\begin{document}
    \graphicspath{{images/}{images2/}} % папки с картинками

    \worknumber{3}
    \variant{4}
    \workname{Русский язык в \LaTeX}
    \discipline{Автоматизация процессов разработки научно-технической документации}
    \group{ИУ6-64Б}
    \author{М.А.Гейне}
    \tutor[Преподаватель]{Т.А.Ким}
    \bmstutitlelab

    \textbf{Цель работы:} 
    получить навыки по использованию LaTeX как инструмента 
    для получения текстов, соответствующих русской типографской традиции.

    \section{Задание 1}
    \input{Lab3 Geine task1.tex}

    \section{Код задания 1}
    \VerbatimInput[frame=lines, breaklines=true,
    breakanywhere=true]{'Lab3 Geine task1.tex'}

    \section{Задание 2}
    \subsection{Создание счётчика}
\newcounter{nc}[subsection]
\stepcounter{nc}
Счётчик \Roman{subsection}.\arabic{nc}: Мой текст...

\stepcounter{nc}
Счётчик \Roman{subsection}.\arabic{nc}: Мой текст...

\stepcounter{nc}
Счётчик \Roman{subsection}.\arabic{nc}: Мой текст...

\subsection{Создание окружения}
\newenvironment{myenvironment}[1]{\refstepcounter{nc}
Окружение~\Roman{subsection}.\arabic{nc}.~<<#1>>}{}

\begin{myenvironment}{Созданное окружение 1.}
    Тело окружения 1
\end{myenvironment}

\begin{myenvironment}{Созданное окружение 2.}
    Тело окружения 2
\end{myenvironment}

\subsection{Маркированный список}
\begin{itemize}
    \item Item 1
    \item[$\bigstar$] Item 2
    \begin{itemize}
        \item[$\rightsquigarrow$ ] Item 2.1
        \begin{itemize}
            \item[$\sharp$ ] Item 2.1.1
            \item[$\flat$ ] Item 2.1.2
        \end{itemize}
        \item[$\looparrowright$ ] Item 2.2
    \end{itemize}
\end{itemize}

\subsection{Нумерованный список}
\renewcommand{\labelenumii}{\Alph{enumii}}
\renewcommand{\labelenumiii}{\Roman{enumiii}}
\begin{enumerate}
    \item Item 1
    \item Item 2
    \begin{enumerate}
        \item Item 2.1
        \begin{enumerate}
            \item Item 2.1.1
            \item Item 2.1.2
        \end{enumerate}
        \item Item 2.2
    \end{enumerate}
\end{enumerate}
\end{document}