Адъютант Бонапарте еще не приехал в отряд Мюрата, и сражение
еще не начиналось. В отряде Багратиона ничего не знали об общем ходе
дел, говорили о мире, но не верили в его возможность. Говорили о сражении и тоже не верили и в близость сражения. Багратион, зная Болконского за любимого и доверенного адъютанта, принял его с особенным
начальническим отличием и снисхождением, объяснил ему, что, вероятно, нынче или завтра будет сражение, и предоставил ему полную свободу
находиться при нем во время сражения или в ариергарде наблюдать за
порядком отступления, <<что тоже было очень важно>>.

"--* Впрочем, нынче, вероятно, дела не будет, "--* сказал Багратион, как
бы успокоивая князя Андрея.<<Ежели это один из обыкновенных штабных франтиков, посылаемых для получения крестика, то он и в ариергарде получит награду, а ежели хочет со мной быть, пускай\ldots~пригодится, коли храбрый офицер>>, подумал Багратион. 
Князь Андрей ничего не ответив, попросил позволения князя объехать позицию и узнать расположение войск с тем, чтобы в случае поручения знать, куда ехать.
Дежурный офицер отряда, мужчина красивый, щеголевато одетый и с
алмазным перстнем на указательном пальце, дурно, но охотно говоривший по-французски, вызвался проводить князя Андрея.

Со всех сторон виднелись мокрые, с грустными лицами офицеры,
чего-то как будто искавшие, и солдаты, тащившие из деревни двери,
лавки и заборы.

"--* Вот не можем, князь, избавиться от этого народа, "--* сказал штабофицер, указывая на этих людей. "--* Распускают командиры. А вот здесь,
"--* он указал на раскинутую палатку маркитанта, "--* собьются и сидят.
Нынче утром всех выгнал: посмотрите, опять полна. Надо подъехать,
князь, пугнуть их. Одна минута.

"--* Заедемте, и я возьму у него сыру и булку, "--* сказал князь Андрей,
который не успел еще поесть.

"--* Что ж вы не сказали, князь? Я бы предложил своего хлеба-соли.