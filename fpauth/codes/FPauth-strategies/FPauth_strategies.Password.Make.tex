\subsection{Модуль \ocamlinlinecode{Password.\allowbreak{}Make}}\label{page-FPauth-strategies-module-FPauth+u+strategies-module-Password-module-Make}%
\ocamlinlinecode{Make} создаёт стратегию для предоставленной модели.

\subsubsection{Параметры\label{parameters}}%
\label{page-FPauth-strategies-module-FPauth+u+strategies-module-Password-module-Make-argument-1-M}\ocamlcodefragment{\ocamltag{keyword}{module} \hyperref[page-FPauth-strategies-module-FPauth+u+strategies-module-Password-module-Make-argument-1-M]{\ocamlinlinecode{M}}}\ocamlcodefragment{ : \ocamltag{keyword}{sig}}\begin{ocamlindent}\label{page-FPauth-strategies-module-FPauth+u+strategies-module-Password-module-Make-argument-1-M-type-t}\ocamlcodefragment{\ocamltag{keyword}{type} t}\\
\label{page-FPauth-strategies-module-FPauth+u+strategies-module-Password-module-Make-argument-1-M-val-encrypted+u+password}\ocamlcodefragment{\ocamltag{keyword}{val} encrypted\_\allowbreak{}password : \hyperref[page-FPauth-strategies-module-FPauth+u+strategies-module-Password-module-Make-argument-1-M-type-t]{\ocamlinlinecode{t}} \ocamltag{arrow}{$\rightarrow$} string option}\begin{ocamlindent}\ocamlinlinecode{encrypted\_\allowbreak{}password} - строка с хэшем пароля, на соответствие которой предоставленный пароль будет проверяться. Argon2 используется для верификации.\end{ocamlindent}%
\medbreak
\end{ocamlindent}%
\ocamlcodefragment{\ocamltag{keyword}{end}}\\
\subsubsection{Сигнатура\label{signature}}%
\label{page-FPauth-strategies-module-FPauth+u+strategies-module-Password-module-Make-type-entity}\ocamlcodefragment{\ocamltag{keyword}{type} entity = \hyperref[page-FPauth-strategies-module-FPauth+u+strategies-module-Password-module-Make-argument-1-M-type-t]{\ocamlinlinecode{M.\allowbreak{}t}}}\\
\label{page-FPauth-strategies-module-FPauth+u+strategies-module-Password-module-Make-val-call}\ocamlcodefragment{\ocamltag{keyword}{val} call : 
  \hyperref[xref-unresolved]{\ocamlinlinecode{Dream}}.\allowbreak{}request \ocamltag{arrow}{$\rightarrow$}
  \hyperref[page-FPauth-strategies-module-FPauth+u+strategies-module-Password-module-Make-type-entity]{\ocamlinlinecode{entity}} \ocamltag{arrow}{$\rightarrow$}
  \hyperref[page-FPauth-strategies-module-FPauth+u+strategies-module-Password-module-Make-type-entity]{\ocamlinlinecode{entity}} \hyperref[page-FPauth-core-module-FPauth+u+core-module-Static-module-StratResult-type-t]{\ocamlinlinecode{FPauth\_\allowbreak{}core.\allowbreak{}Static.\allowbreak{}StratResult.\allowbreak{}t}} \hyperref[xref-unresolved]{\ocamlinlinecode{Lwt}}.\allowbreak{}t}\begin{ocamlindent}\ocamlinlinecode{call} является главной функцией стратегии, которая аутентифицирует пользователя по параметру "password". Параметр проверяется на соответствие хэшу пароля, созданному с помошью Argon2.\end{ocamlindent}%
\medbreak
\label{page-FPauth-strategies-module-FPauth+u+strategies-module-Password-module-Make-val-routes}\ocamlcodefragment{\ocamltag{keyword}{val} routes : \hyperref[xref-unresolved]{\ocamlinlinecode{Dream}}.\allowbreak{}route}\begin{ocamlindent}Эта стратегия не имеет маршрутов и возвращает \ocamlinlinecode{Dream.\allowbreak{}no\_\allowbreak{}route}.\end{ocamlindent}%
\medbreak
\label{page-FPauth-strategies-module-FPauth+u+strategies-module-Password-module-Make-val-name}\ocamlcodefragment{\ocamltag{keyword}{val} name : string}\begin{ocamlindent}См. \hyperref[page-FPauth-strategies-module-FPauth+u+strategies-module-Password-val-name]{\ocamlinlinecode{\ocamlinlinecode{Password.\allowbreak{}name}}}\end{ocamlindent}%
\medbreak


