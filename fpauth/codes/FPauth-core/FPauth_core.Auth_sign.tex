\subsection{Модуль \ocamlinlinecode{FPauth\_\allowbreak{}core.\allowbreak{}Auth\_\allowbreak{}sign}}\label{page-FPauth-core-module-FPauth+u+core-module-Auth+u+sign}%
\ocamlinlinecode{Auth\_\allowbreak{}sign} - модуль, содержащий сигнатуры других модулей, которые могут быть реализованы и интегрированы извне библиотеки, а также сигнатуры некоторых внутренних модулей.

В этом модуле хранятся сигнатуры модулей системы аутентификации или модулей, которые необходимо реализовать извне.

\label{page-FPauth-core-module-FPauth+u+core-module-Auth+u+sign-module-type-MODEL}\ocamlcodefragment{\ocamltag{keyword}{module} \ocamltag{keyword}{type} \hyperref[page-FPauth-core-module-FPauth+u+core-module-Auth+u+sign-module-type-MODEL]{\ocamlinlinecode{MODEL}}}\ocamlcodefragment{ = \ocamltag{keyword}{sig}}\begin{ocamlindent}\label{page-FPauth-core-module-FPauth+u+core-module-Auth+u+sign-module-type-MODEL-type-t}\ocamlcodefragment{\ocamltag{keyword}{type} t}\begin{ocamlindent}Некоторое представление сущности, которая будет аутентифицирована.\end{ocamlindent}%
\medbreak
\label{page-FPauth-core-module-FPauth+u+core-module-Auth+u+sign-module-type-MODEL-val-serialize}\ocamlcodefragment{\ocamltag{keyword}{val} serialize : \hyperref[page-FPauth-core-module-FPauth+u+core-module-Auth+u+sign-module-type-MODEL-type-t]{\ocamlinlinecode{t}} \ocamltag{arrow}{$\rightarrow$} \hyperref[xref-unresolved]{\ocamlinlinecode{Base}}.\allowbreak{}string}\begin{ocamlindent}\ocamlinlinecode{serialize ent} создаёт \ocamlinlinecode{string} из \hyperref[page-FPauth-core-module-FPauth+u+core-module-Auth+u+sign-module-type-MODEL-type-t]{\ocamlinlinecode{\ocamlinlinecode{t}}}.\end{ocamlindent}%
\medbreak
\label{page-FPauth-core-module-FPauth+u+core-module-Auth+u+sign-module-type-MODEL-val-deserialize}\ocamlcodefragment{\ocamltag{keyword}{val} deserialize : \hyperref[xref-unresolved]{\ocamlinlinecode{Base}}.\allowbreak{}string \ocamltag{arrow}{$\rightarrow$} ( \hyperref[page-FPauth-core-module-FPauth+u+core-module-Auth+u+sign-module-type-MODEL-type-t]{\ocamlinlinecode{t}},\allowbreak{} \hyperref[xref-unresolved]{\ocamlinlinecode{Base}}.\allowbreak{}Error.\allowbreak{}t ) \hyperref[xref-unresolved]{\ocamlinlinecode{Base}}.\allowbreak{}Result.\allowbreak{}t}\begin{ocamlindent}\ocamlinlinecode{deserialize} создаёт \hyperref[page-FPauth-core-module-FPauth+u+core-module-Auth+u+sign-module-type-MODEL-type-t]{\ocamlinlinecode{\ocamlinlinecode{t}}}. Возвращает: \ocamlinlinecode{Ok t} если десериализация была успешна или \ocamlinlinecode{Error string} если произошла ошибка.\end{ocamlindent}%
\medbreak
\label{page-FPauth-core-module-FPauth+u+core-module-Auth+u+sign-module-type-MODEL-val-identificate}\ocamlcodefragment{\ocamltag{keyword}{val} identificate : 
  \hyperref[xref-unresolved]{\ocamlinlinecode{Dream}}.\allowbreak{}request \ocamltag{arrow}{$\rightarrow$}
  ( \hyperref[page-FPauth-core-module-FPauth+u+core-module-Auth+u+sign-module-type-MODEL-type-t]{\ocamlinlinecode{t}},\allowbreak{} \hyperref[xref-unresolved]{\ocamlinlinecode{Base}}.\allowbreak{}Error.\allowbreak{}t ) \hyperref[xref-unresolved]{\ocamlinlinecode{Base}}.\allowbreak{}Result.\allowbreak{}t \hyperref[xref-unresolved]{\ocamlinlinecode{Dream}}.\allowbreak{}promise}\begin{ocamlindent}\ocamlinlinecode{identificate} определяет, какая именно сущность аутентифицируется. Находит репрезентацию сущности или возвращает ошибку.\end{ocamlindent}%
\medbreak
\label{page-FPauth-core-module-FPauth+u+core-module-Auth+u+sign-module-type-MODEL-val-applicable+u+strats}\ocamlcodefragment{\ocamltag{keyword}{val} applicable\_\allowbreak{}strats : \hyperref[page-FPauth-core-module-FPauth+u+core-module-Auth+u+sign-module-type-MODEL-type-t]{\ocamlinlinecode{t}} \ocamltag{arrow}{$\rightarrow$} \hyperref[xref-unresolved]{\ocamlinlinecode{Base}}.\allowbreak{}string \hyperref[xref-unresolved]{\ocamlinlinecode{Base}}.\allowbreak{}list}\begin{ocamlindent}\ocamlinlinecode{applicable\_\allowbreak{}strats} возвращает список стратегий, которые могут быть применены ко всей \hyperref[page-FPauth-core-module-FPauth+u+core-module-Auth+u+sign-module-type-MODEL]{\ocamlinlinecode{\ocamlinlinecode{MODEL}}}.\end{ocamlindent}%
\medbreak
\end{ocamlindent}%
\ocamlcodefragment{\ocamltag{keyword}{end}}\begin{ocamlindent}\ocamlinlinecode{MODEL} - сигнатура модулей, обслуживающих аутентифицируемые сущности.\end{ocamlindent}%
\medbreak
\label{page-FPauth-core-module-FPauth+u+core-module-Auth+u+sign-module-type-SESSIONMANAGER}\ocamlcodefragment{\ocamltag{keyword}{module} \ocamltag{keyword}{type} \hyperref[page-FPauth-core-module-FPauth+u+core-module-Auth+u+sign-module-type-SESSIONMANAGER]{\ocamlinlinecode{SESSIONMANAGER}}}\ocamlcodefragment{ = \ocamltag{keyword}{sig}}\begin{ocamlindent}\label{page-FPauth-core-module-FPauth+u+core-module-Auth+u+sign-module-type-SESSIONMANAGER-type-entity}\ocamlcodefragment{\ocamltag{keyword}{type} entity}\begin{ocamlindent}тип \ocamlinlinecode{entity} - тип аутентифицируемой сущности, совпадающий с \hyperref[page-FPauth-core-module-FPauth+u+core-module-Auth+u+sign-module-type-MODEL-type-t]{\ocamlinlinecode{\ocamlinlinecode{MODEL.\allowbreak{}t}}}\end{ocamlindent}%
\medbreak
\label{page-FPauth-core-module-FPauth+u+core-module-Auth+u+sign-module-type-SESSIONMANAGER-val-auth+u+setup}\ocamlcodefragment{\ocamltag{keyword}{val} auth\_\allowbreak{}setup : \hyperref[xref-unresolved]{\ocamlinlinecode{Dream}}.\allowbreak{}middleware}\begin{ocamlindent}\ocamlinlinecode{auth\_\allowbreak{}setup} - middleware, которое контролирует сессию, устанавливает переменные-\ocamlinlinecode{field} и всопмогательные функции для последующих handlers.\ocamlinlinecode{auth\_\allowbreak{}setup} пробует извлечь строку из сессии с ключом \ocamlinlinecode{auth} и определить статус аутентификации. Если поле \ocamlinlinecode{auth} отсутствует, то аутентификация не была пройдена. Если \ocamlinlinecode{auth} имеется, то по строке проверяется и изменяется статус аутентификации:\begin{itemize}\item{Если \ocamlinlinecode{auth} содержит пустую строку, то ситуация считается ошибочной;}%
\item{Если \ocamlinlinecode{M}.deserialize вернула \ocamlinlinecode{Error Error.\allowbreak{}t}, то аутентификация не пройдена и ситуация считается ошибочной;}%
\item{Если \ocamlinlinecode{M}.deserialize вернула \ocamlinlinecode{Ok M.\allowbreak{}t}, то аутентификация считается успешной и устанавливаются \hyperref[page-FPauth-core-module-FPauth+u+core-module-Auth+u+sign-module-type-VARIABLES-val-current+u+user]{\ocamlinlinecode{\ocamlinlinecode{VARIABLES.\allowbreak{}current\_\allowbreak{}user}}}. Если с сессией что-то не так, то сессия становится недействительной, ошибка записывается и статус 401 отправляется. Если сессия в порядке, то запрос поступает в следующий обработчик.}\end{itemize}%
\end{ocamlindent}%
\medbreak
\end{ocamlindent}%
\ocamlcodefragment{\ocamltag{keyword}{end}}\begin{ocamlindent}\ocamlinlinecode{SESSIONMANAGER} - сигнатура для функтора, создающего модуль контроллера сессий для аутентификации сущностей типа \hyperref[page-FPauth-core-module-FPauth+u+core-module-Auth+u+sign-module-type-MODEL-type-t]{\ocamlinlinecode{\ocamlinlinecode{MODEL.\allowbreak{}t}}}.\end{ocamlindent}%
\medbreak
\label{page-FPauth-core-module-FPauth+u+core-module-Auth+u+sign-module-type-STRATEGY}\ocamlcodefragment{\ocamltag{keyword}{module} \ocamltag{keyword}{type} \hyperref[page-FPauth-core-module-FPauth+u+core-module-Auth+u+sign-module-type-STRATEGY]{\ocamlinlinecode{STRATEGY}}}\ocamlcodefragment{ = \ocamltag{keyword}{sig}}\begin{ocamlindent}\label{page-FPauth-core-module-FPauth+u+core-module-Auth+u+sign-module-type-STRATEGY-type-entity}\ocamlcodefragment{\ocamltag{keyword}{type} entity}\begin{ocamlindent}тип \ocamlinlinecode{entity} - тип аутентифицируемой сущности, совпадающий с \hyperref[page-FPauth-core-module-FPauth+u+core-module-Auth+u+sign-module-type-MODEL-type-t]{\ocamlinlinecode{\ocamlinlinecode{MODEL.\allowbreak{}t}}}\end{ocamlindent}%
\medbreak
\label{page-FPauth-core-module-FPauth+u+core-module-Auth+u+sign-module-type-STRATEGY-val-call}\ocamlcodefragment{\ocamltag{keyword}{val} call : \hyperref[xref-unresolved]{\ocamlinlinecode{Dream}}.\allowbreak{}request \ocamltag{arrow}{$\rightarrow$} \hyperref[page-FPauth-core-module-FPauth+u+core-module-Auth+u+sign-module-type-STRATEGY-type-entity]{\ocamlinlinecode{entity}} \ocamltag{arrow}{$\rightarrow$} \hyperref[page-FPauth-core-module-FPauth+u+core-module-Auth+u+sign-module-type-STRATEGY-type-entity]{\ocamlinlinecode{entity}} \hyperref[page-FPauth-core-module-FPauth+u+core-module-Static-module-StratResult-type-t]{\ocamlinlinecode{Static.\allowbreak{}StratResult.\allowbreak{}t}} \hyperref[xref-unresolved]{\ocamlinlinecode{Lwt}}.\allowbreak{}t}\begin{ocamlindent}\ocamlinlinecode{call request entity} является основной функцией стратегии. Является непосредственным методом подтверждения личности.\end{ocamlindent}%
\medbreak
\label{page-FPauth-core-module-FPauth+u+core-module-Auth+u+sign-module-type-STRATEGY-val-routes}\ocamlcodefragment{\ocamltag{keyword}{val} routes : \hyperref[xref-unresolved]{\ocamlinlinecode{Dream}}.\allowbreak{}route}\begin{ocamlindent}\ocamlinlinecode{routes} определяет дополнительные маршруты к handlers стратегии, если они необходимы. Может включать несколько маршрутов, если применить \ocamlinlinecode{Dream.\allowbreak{}scope}.\end{ocamlindent}%
\medbreak
\label{page-FPauth-core-module-FPauth+u+core-module-Auth+u+sign-module-type-STRATEGY-val-name}\ocamlcodefragment{\ocamltag{keyword}{val} name : \hyperref[xref-unresolved]{\ocamlinlinecode{Base}}.\allowbreak{}string}\begin{ocamlindent}\ocamlinlinecode{name} - имя стратегии. Используется для определния, может ли страиегия быть применена к определённой сущности.\end{ocamlindent}%
\medbreak
\end{ocamlindent}%
\ocamlcodefragment{\ocamltag{keyword}{end}}\begin{ocamlindent}\ocamlinlinecode{STRATEGY} - модуль, который содержит функции для аутентификации сущности определённым способом, а также дополнительные маршруты и функции.\end{ocamlindent}%
\medbreak
\label{page-FPauth-core-module-FPauth+u+core-module-Auth+u+sign-module-type-AUTHENTICATOR}\ocamlcodefragment{\ocamltag{keyword}{module} \ocamltag{keyword}{type} \hyperref[page-FPauth-core-module-FPauth+u+core-module-Auth+u+sign-module-type-AUTHENTICATOR]{\ocamlinlinecode{AUTHENTICATOR}}}\ocamlcodefragment{ = \ocamltag{keyword}{sig}}\begin{ocamlindent}\label{page-FPauth-core-module-FPauth+u+core-module-Auth+u+sign-module-type-AUTHENTICATOR-type-entity}\ocamlcodefragment{\ocamltag{keyword}{type} entity}\begin{ocamlindent}тип \ocamlinlinecode{entity} - тип аутентифицируемой сущности, совпадающий с \hyperref[page-FPauth-core-module-FPauth+u+core-module-Auth+u+sign-module-type-MODEL-type-t]{\ocamlinlinecode{\ocamlinlinecode{MODEL.\allowbreak{}t}}}.\end{ocamlindent}%
\medbreak
\label{page-FPauth-core-module-FPauth+u+core-module-Auth+u+sign-module-type-AUTHENTICATOR-type-strategy}\ocamlcodefragment{\ocamltag{keyword}{type} strategy = (\ocamltag{keyword}{module} \hyperref[page-FPauth-core-module-FPauth+u+core-module-Auth+u+sign-module-type-STRATEGY]{\ocamlinlinecode{STRATEGY}} \ocamltag{keyword}{with} \ocamltag{keyword}{type} \hyperref[page-FPauth-core-module-FPauth+u+core-module-Auth+u+sign-module-type-STRATEGY-type-entity]{\ocamlinlinecode{entity}} = \hyperref[page-FPauth-core-module-FPauth+u+core-module-Auth+u+sign-module-type-AUTHENTICATOR-type-entity]{\ocamlinlinecode{entity}})}\begin{ocamlindent}\ocamlinlinecode{strategy} - модуль первого класса стратегии для \hyperref[page-FPauth-core-module-FPauth+u+core-module-Auth+u+sign-module-type-AUTHENTICATOR-type-entity]{\ocamlinlinecode{\ocamlinlinecode{entity}}}.\end{ocamlindent}%
\medbreak
\label{page-FPauth-core-module-FPauth+u+core-module-Auth+u+sign-module-type-AUTHENTICATOR-val-authenticate}\ocamlcodefragment{\ocamltag{keyword}{val} authenticate : 
  \hyperref[page-FPauth-core-module-FPauth+u+core-module-Auth+u+sign-module-type-AUTHENTICATOR-type-strategy]{\ocamlinlinecode{strategy}} \hyperref[xref-unresolved]{\ocamlinlinecode{Base}}.\allowbreak{}list \ocamltag{arrow}{$\rightarrow$}
  \hyperref[xref-unresolved]{\ocamlinlinecode{Dream}}.\allowbreak{}request \ocamltag{arrow}{$\rightarrow$}
  \hyperref[page-FPauth-core-module-FPauth+u+core-module-Static-module-AuthResult-type-t]{\ocamlinlinecode{Static.\allowbreak{}AuthResult.\allowbreak{}t}} \hyperref[xref-unresolved]{\ocamlinlinecode{Dream}}.\allowbreak{}promise}\begin{ocamlindent}\ocamlinlinecode{authenticate} запускается множество стратегий для запроса и определяет, была ли аутентификация в целом успешной или нет.\end{ocamlindent}%
\medbreak
\label{page-FPauth-core-module-FPauth+u+core-module-Auth+u+sign-module-type-AUTHENTICATOR-val-logout}\ocamlcodefragment{\ocamltag{keyword}{val} logout : \hyperref[xref-unresolved]{\ocamlinlinecode{Dream}}.\allowbreak{}request \ocamltag{arrow}{$\rightarrow$} \hyperref[xref-unresolved]{\ocamlinlinecode{Base}}.\allowbreak{}unit \hyperref[xref-unresolved]{\ocamlinlinecode{Lwt}}.\allowbreak{}t}\begin{ocamlindent}\ocamlinlinecode{logout} сбрасывает сессию, что приводит к сбросу статуса аутентификации. В связи с особенностями работы \ocamlinlinecode{field} текущий пользователь будет сброшен только в следующем запросе.\end{ocamlindent}%
\medbreak
\end{ocamlindent}%
\ocamlcodefragment{\ocamltag{keyword}{end}}\begin{ocamlindent}\ocamlinlinecode{AUTHENTICATOR} - сигнатура для функтора, создающего аутентификаторы для различных сущностей.\end{ocamlindent}%
\medbreak
\label{page-FPauth-core-module-FPauth+u+core-module-Auth+u+sign-module-type-VARIABLES}\ocamlcodefragment{\ocamltag{keyword}{module} \ocamltag{keyword}{type} \hyperref[page-FPauth-core-module-FPauth+u+core-module-Auth+u+sign-module-type-VARIABLES]{\ocamlinlinecode{VARIABLES}}}\ocamlcodefragment{ = \ocamltag{keyword}{sig}}\begin{ocamlindent}\label{page-FPauth-core-module-FPauth+u+core-module-Auth+u+sign-module-type-VARIABLES-type-entity}\ocamlcodefragment{\ocamltag{keyword}{type} entity}\begin{ocamlindent}тип \ocamlinlinecode{entity} - тип аутентифицируемой сущности, совпадающий с \hyperref[page-FPauth-core-module-FPauth+u+core-module-Auth+u+sign-module-type-MODEL-type-t]{\ocamlinlinecode{\ocamlinlinecode{MODEL.\allowbreak{}t}}}.\end{ocamlindent}%
\medbreak
\label{page-FPauth-core-module-FPauth+u+core-module-Auth+u+sign-module-type-VARIABLES-val-authenticated}\ocamlcodefragment{\ocamltag{keyword}{val} authenticated : \hyperref[xref-unresolved]{\ocamlinlinecode{Base}}.\allowbreak{}bool \hyperref[xref-unresolved]{\ocamlinlinecode{Dream}}.\allowbreak{}field}\begin{ocamlindent}\ocamlinlinecode{authenticated} - переменная, действительная в рамках одного запроса, отражает, была ли пройдена аутентификация ранее. Устанавливается в \hyperref[page-FPauth-core-module-FPauth+u+core-module-Auth+u+sign-module-type-SESSIONMANAGER-val-auth+u+setup]{\ocamlinlinecode{\ocamlinlinecode{SESSIONMANAGER.\allowbreak{}auth\_\allowbreak{}setup}}}.\end{ocamlindent}%
\medbreak
\label{page-FPauth-core-module-FPauth+u+core-module-Auth+u+sign-module-type-VARIABLES-val-current+u+user}\ocamlcodefragment{\ocamltag{keyword}{val} current\_\allowbreak{}user : \hyperref[page-FPauth-core-module-FPauth+u+core-module-Auth+u+sign-module-type-VARIABLES-type-entity]{\ocamlinlinecode{entity}} \hyperref[xref-unresolved]{\ocamlinlinecode{Dream}}.\allowbreak{}field}\begin{ocamlindent}\ocamlinlinecode{current\_\allowbreak{}user} - переменная, действительная в рамках одного запроса, содержит аутентифицированную сущность (если ранее была пройдена аутентификация). Устанавливается в \hyperref[page-FPauth-core-module-FPauth+u+core-module-Auth+u+sign-module-type-SESSIONMANAGER-val-auth+u+setup]{\ocamlinlinecode{\ocamlinlinecode{SESSIONMANAGER.\allowbreak{}auth\_\allowbreak{}setup}}}\end{ocamlindent}%
\medbreak
\label{page-FPauth-core-module-FPauth+u+core-module-Auth+u+sign-module-type-VARIABLES-val-auth+u+error}\ocamlcodefragment{\ocamltag{keyword}{val} auth\_\allowbreak{}error : \hyperref[xref-unresolved]{\ocamlinlinecode{Base}}.\allowbreak{}Error.\allowbreak{}t \hyperref[xref-unresolved]{\ocamlinlinecode{Dream}}.\allowbreak{}field}\begin{ocamlindent}\ocamlinlinecode{auth\_\allowbreak{}error} - field-переменная с ошибкой, которая могла произойти на любом этапе аутентификации. Устанавливается в \hyperref[page-FPauth-core-module-FPauth+u+core-module-Auth+u+sign-module-type-AUTHENTICATOR-val-authenticate]{\ocamlinlinecode{\ocamlinlinecode{AUTHENTICATOR.\allowbreak{}authenticate}}}.\end{ocamlindent}%
\medbreak
\label{page-FPauth-core-module-FPauth+u+core-module-Auth+u+sign-module-type-VARIABLES-val-update+u+current+u+user}\ocamlcodefragment{\ocamltag{keyword}{val} update\_\allowbreak{}current\_\allowbreak{}user : \hyperref[page-FPauth-core-module-FPauth+u+core-module-Auth+u+sign-module-type-VARIABLES-type-entity]{\ocamlinlinecode{entity}} \ocamltag{arrow}{$\rightarrow$} \hyperref[xref-unresolved]{\ocamlinlinecode{Dream}}.\allowbreak{}request \ocamltag{arrow}{$\rightarrow$} \hyperref[xref-unresolved]{\ocamlinlinecode{Base}}.\allowbreak{}unit \hyperref[xref-unresolved]{\ocamlinlinecode{Dream}}.\allowbreak{}promise}\begin{ocamlindent}\ocamlinlinecode{update\_\allowbreak{}current\_\allowbreak{}user user request} обновляет \hyperref[page-FPauth-core-module-FPauth+u+core-module-Auth+u+sign-module-type-VARIABLES-val-current+u+user]{\ocamlinlinecode{\ocamlinlinecode{current\_\allowbreak{}user}}} и сессию. Необходимо использовать в том случе, если были внесены изменения, влияющие на сериализацию.\end{ocamlindent}%
\medbreak
\end{ocamlindent}%
\ocamlcodefragment{\ocamltag{keyword}{end}}\begin{ocamlindent}\ocamlinlinecode{VARIABLES} - сигнатура модуля, содержащего field-переменные, основанные на \hyperref[page-FPauth-core-module-FPauth+u+core-module-Auth+u+sign-module-type-MODEL]{\ocamlinlinecode{\ocamlinlinecode{MODEL}}}.\end{ocamlindent}%
\medbreak
\label{page-FPauth-core-module-FPauth+u+core-module-Auth+u+sign-module-type-RESPONSES}\ocamlcodefragment{\ocamltag{keyword}{module} \ocamltag{keyword}{type} \hyperref[page-FPauth-core-module-FPauth+u+core-module-Auth+u+sign-module-type-RESPONSES]{\ocamlinlinecode{RESPONSES}}}\ocamlcodefragment{ = \ocamltag{keyword}{sig}}\begin{ocamlindent}\label{page-FPauth-core-module-FPauth+u+core-module-Auth+u+sign-module-type-RESPONSES-val-login+u+successful}\ocamlcodefragment{\ocamltag{keyword}{val} login\_\allowbreak{}successful : \hyperref[xref-unresolved]{\ocamlinlinecode{Dream}}.\allowbreak{}request \ocamltag{arrow}{$\rightarrow$} \hyperref[xref-unresolved]{\ocamlinlinecode{Dream}}.\allowbreak{}response \hyperref[xref-unresolved]{\ocamlinlinecode{Dream}}.\allowbreak{}promise}\begin{ocamlindent}\ocamlinlinecode{login\_\allowbreak{}successful} вызывается в случае, если аутентификация была успешна.\end{ocamlindent}%
\medbreak
\label{page-FPauth-core-module-FPauth+u+core-module-Auth+u+sign-module-type-RESPONSES-val-login+u+error}\ocamlcodefragment{\ocamltag{keyword}{val} login\_\allowbreak{}error : \hyperref[xref-unresolved]{\ocamlinlinecode{Dream}}.\allowbreak{}request \ocamltag{arrow}{$\rightarrow$} \hyperref[xref-unresolved]{\ocamlinlinecode{Dream}}.\allowbreak{}response \hyperref[xref-unresolved]{\ocamlinlinecode{Dream}}.\allowbreak{}promise}\begin{ocamlindent}\ocamlinlinecode{login\_\allowbreak{}error} вызывается в случае, если в рамках аутентификации произошла ошибка.\end{ocamlindent}%
\medbreak
\label{page-FPauth-core-module-FPauth+u+core-module-Auth+u+sign-module-type-RESPONSES-val-logout}\ocamlcodefragment{\ocamltag{keyword}{val} logout : \hyperref[xref-unresolved]{\ocamlinlinecode{Dream}}.\allowbreak{}request \ocamltag{arrow}{$\rightarrow$} \hyperref[xref-unresolved]{\ocamlinlinecode{Dream}}.\allowbreak{}response \hyperref[xref-unresolved]{\ocamlinlinecode{Dream}}.\allowbreak{}promise}\begin{ocamlindent}\ocamlinlinecode{logout} вызывается после того, как аутентификация была сброшена.\end{ocamlindent}%
\medbreak
\end{ocamlindent}%
\ocamlcodefragment{\ocamltag{keyword}{end}}\begin{ocamlindent}\ocamlinlinecode{RESPONSES} - сигнатура модуля, определяющего способ представления базовых событий библиотеки.\end{ocamlindent}%
\medbreak
\label{page-FPauth-core-module-FPauth+u+core-module-Auth+u+sign-module-type-ROUTER}\ocamlcodefragment{\ocamltag{keyword}{module} \ocamltag{keyword}{type} \hyperref[page-FPauth-core-module-FPauth+u+core-module-Auth+u+sign-module-type-ROUTER]{\ocamlinlinecode{ROUTER}}}\ocamlcodefragment{ = \ocamltag{keyword}{sig}}\begin{ocamlindent}\label{page-FPauth-core-module-FPauth+u+core-module-Auth+u+sign-module-type-ROUTER-type-entity}\ocamlcodefragment{\ocamltag{keyword}{type} entity}\begin{ocamlindent}тип \ocamlinlinecode{entity} - тип аутентифицируемой сущности, совпадающий с \hyperref[page-FPauth-core-module-FPauth+u+core-module-Auth+u+sign-module-type-MODEL-type-t]{\ocamlinlinecode{\ocamlinlinecode{MODEL.\allowbreak{}t}}}.\end{ocamlindent}%
\medbreak
\label{page-FPauth-core-module-FPauth+u+core-module-Auth+u+sign-module-type-ROUTER-type-strategy}\ocamlcodefragment{\ocamltag{keyword}{type} strategy = (\ocamltag{keyword}{module} \hyperref[page-FPauth-core-module-FPauth+u+core-module-Auth+u+sign-module-type-STRATEGY]{\ocamlinlinecode{STRATEGY}} \ocamltag{keyword}{with} \ocamltag{keyword}{type} \hyperref[page-FPauth-core-module-FPauth+u+core-module-Auth+u+sign-module-type-STRATEGY-type-entity]{\ocamlinlinecode{entity}} = \hyperref[page-FPauth-core-module-FPauth+u+core-module-Auth+u+sign-module-type-ROUTER-type-entity]{\ocamlinlinecode{entity}})}\begin{ocamlindent}\ocamlinlinecode{strategy} - модуль первого класса стратегии для \hyperref[page-FPauth-core-module-FPauth+u+core-module-Auth+u+sign-module-type-ROUTER-type-entity]{\ocamlinlinecode{\ocamlinlinecode{entity}}}.\end{ocamlindent}%
\medbreak
\label{page-FPauth-core-module-FPauth+u+core-module-Auth+u+sign-module-type-ROUTER-val-login+u+handler}\ocamlcodefragment{\ocamltag{keyword}{val} login\_\allowbreak{}handler : 
  \hyperref[page-FPauth-core-module-FPauth+u+core-module-Auth+u+sign-module-type-ROUTER-type-strategy]{\ocamlinlinecode{strategy}} \hyperref[xref-unresolved]{\ocamlinlinecode{Base}}.\allowbreak{}list \ocamltag{arrow}{$\rightarrow$}
  (\ocamltag{keyword}{module} \hyperref[page-FPauth-core-module-FPauth+u+core-module-Auth+u+sign-module-type-RESPONSES]{\ocamlinlinecode{RESPONSES}}) \ocamltag{arrow}{$\rightarrow$}
  \hyperref[xref-unresolved]{\ocamlinlinecode{Dream}}.\allowbreak{}request \ocamltag{arrow}{$\rightarrow$}
  \hyperref[xref-unresolved]{\ocamlinlinecode{Dream}}.\allowbreak{}response \hyperref[xref-unresolved]{\ocamlinlinecode{Lwt}}.\allowbreak{}t}\begin{ocamlindent}\ocamlinlinecode{login\_\allowbreak{}handler} получается список стратегий и шаблоны ответов, запускает аутентификацию и обрабатывает её результаты.\end{ocamlindent}%
\medbreak
\label{page-FPauth-core-module-FPauth+u+core-module-Auth+u+sign-module-type-ROUTER-val-logout+u+handler}\ocamlcodefragment{\ocamltag{keyword}{val} logout\_\allowbreak{}handler : 
  (\ocamltag{keyword}{module} \hyperref[page-FPauth-core-module-FPauth+u+core-module-Auth+u+sign-module-type-RESPONSES]{\ocamlinlinecode{RESPONSES}}) \ocamltag{arrow}{$\rightarrow$}
  \hyperref[xref-unresolved]{\ocamlinlinecode{Dream}}.\allowbreak{}request \ocamltag{arrow}{$\rightarrow$}
  \hyperref[xref-unresolved]{\ocamlinlinecode{Dream}}.\allowbreak{}response \hyperref[xref-unresolved]{\ocamlinlinecode{Lwt}}.\allowbreak{}t}\begin{ocamlindent}\ocamlinlinecode{logout\_\allowbreak{}handler} сбрасывает аутентификацию для текущего пользователя.\end{ocamlindent}%
\medbreak
\label{page-FPauth-core-module-FPauth+u+core-module-Auth+u+sign-module-type-ROUTER-val-call}\ocamlcodefragment{\ocamltag{keyword}{val} call : 
  ?root:\hyperref[xref-unresolved]{\ocamlinlinecode{Base}}.\allowbreak{}string \ocamltag{arrow}{$\rightarrow$}
  responses:(\ocamltag{keyword}{module} \hyperref[page-FPauth-core-module-FPauth+u+core-module-Auth+u+sign-module-type-RESPONSES]{\ocamlinlinecode{RESPONSES}}) \ocamltag{arrow}{$\rightarrow$}
  extractor:\hyperref[page-FPauth-core-module-FPauth+u+core-module-Static-module-Params-type-extractor]{\ocamlinlinecode{Static.\allowbreak{}Params.\allowbreak{}extractor}} \ocamltag{arrow}{$\rightarrow$}
  \hyperref[page-FPauth-core-module-FPauth+u+core-module-Auth+u+sign-module-type-ROUTER-type-strategy]{\ocamlinlinecode{strategy}} \hyperref[xref-unresolved]{\ocamlinlinecode{Base}}.\allowbreak{}list \ocamltag{arrow}{$\rightarrow$}
  \hyperref[xref-unresolved]{\ocamlinlinecode{Dream}}.\allowbreak{}route}\begin{ocamlindent}\ocamlinlinecode{call ?root \textasciitilde{}responses \textasciitilde{}extractor strat\_\allowbreak{}list} создаёт маршруты для аутентификации, которые добавляются в \ocamlinlinecode{Dream.\allowbreak{}router}.Содержит следующие базовые маршруты:\begin{itemize}\item{"/auth" является стартовой точкой для аутентификации. Передаёт \ocamlinlinecode{strategies} в \hyperref[page-FPauth-core-module-FPauth+u+core-module-Auth+u+sign-module-type-AUTHENTICATOR-val-authenticate]{\ocamlinlinecode{\ocamlinlinecode{Auth\_\allowbreak{}sign.\allowbreak{}AUTHENTICATOR.\allowbreak{}authenticate}}}.}%
\item{"/logout" выаолняет сброс аутентификации с помощью \ocamlinlinecode{Authenticator}.logout и отвечает с использованием шаблона \hyperref[page-FPauth-core-module-FPauth+u+core-module-Auth+u+sign-module-type-RESPONSES-val-logout]{\ocamlinlinecode{\ocamlinlinecode{Auth\_\allowbreak{}sign.\allowbreak{}RESPONSES.\allowbreak{}logout}}}.}\end{itemize}%
\ocamlinlinecode{extractor} определяет способ извлечения параметров из запросов для всех запросов, связанных с аутентификацией, в том числе поступающих по маршрутам \hyperref[page-FPauth-core-module-FPauth+u+core-module-Auth+u+sign-module-type-STRATEGY-val-routes]{\ocamlinlinecode{\ocamlinlinecode{STRATEGY.\allowbreak{}routes}}}.\ocamlinlinecode{responses} определяет, какие ответы отправлять дял запросов, поступающих по базовым маршрутам.\ocamlinlinecode{?root} определяет корневой путь для всех маршрутов, связанных с аутентификацией. По умолчанию "/".\end{ocamlindent}%
\medbreak
\end{ocamlindent}%
\ocamlcodefragment{\ocamltag{keyword}{end}}\begin{ocamlindent}\ocamlinlinecode{ROUTER} - сигнатура модуля, который содержит handlers для аутентификации и создаёт для них маршруты.\end{ocamlindent}%
\medbreak


