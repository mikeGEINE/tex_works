\subsection{Модуль \ocamlinlinecode{Router.\allowbreak{}Make}}\label{page-FPauth-core-module-FPauth+u+core-module-Router-module-Make}%
\ocamlinlinecode{Make} создаёт экземпляр \hyperref[page-FPauth-core-module-FPauth+u+core-module-Auth+u+sign-module-type-ROUTER]{\ocamlinlinecode{\ocamlinlinecode{Auth\_\allowbreak{}sign.\allowbreak{}ROUTER}}} со всеми его зависимостями.

\subsubsection{Параметры\label{parameters}}%
\label{page-FPauth-core-module-FPauth+u+core-module-Router-module-Make-argument-1-M}\ocamlcodefragment{\ocamltag{keyword}{module} \hyperref[page-FPauth-core-module-FPauth+u+core-module-Router-module-Make-argument-1-M]{\ocamlinlinecode{M}}}\ocamlcodefragment{ : \ocamltag{keyword}{sig}}\begin{ocamlindent}\label{page-FPauth-core-module-FPauth+u+core-module-Router-module-Make-argument-1-M-type-t}\ocamlcodefragment{\ocamltag{keyword}{type} t}\begin{ocamlindent}Некоторое представление сущности, которая будет аутентифицирована.\end{ocamlindent}%
\medbreak
\label{page-FPauth-core-module-FPauth+u+core-module-Router-module-Make-argument-1-M-val-serialize}\ocamlcodefragment{\ocamltag{keyword}{val} serialize : \hyperref[page-FPauth-core-module-FPauth+u+core-module-Router-module-Make-argument-1-M-type-t]{\ocamlinlinecode{t}} \ocamltag{arrow}{$\rightarrow$} \hyperref[xref-unresolved]{\ocamlinlinecode{Base}}.\allowbreak{}string}\begin{ocamlindent}\ocamlinlinecode{serialize ent} создаёт \ocamlinlinecode{string} из \hyperref[page-FPauth-core-module-FPauth+u+core-module-Router-module-Make-argument-1-M-type-t]{\ocamlinlinecode{\ocamlinlinecode{t}}}.\end{ocamlindent}%
\medbreak
\label{page-FPauth-core-module-FPauth+u+core-module-Router-module-Make-argument-1-M-val-deserialize}\ocamlcodefragment{\ocamltag{keyword}{val} deserialize : \hyperref[xref-unresolved]{\ocamlinlinecode{Base}}.\allowbreak{}string \ocamltag{arrow}{$\rightarrow$} ( \hyperref[page-FPauth-core-module-FPauth+u+core-module-Router-module-Make-argument-1-M-type-t]{\ocamlinlinecode{t}},\allowbreak{} \hyperref[xref-unresolved]{\ocamlinlinecode{Base}}.\allowbreak{}Error.\allowbreak{}t ) \hyperref[xref-unresolved]{\ocamlinlinecode{Base}}.\allowbreak{}Result.\allowbreak{}t}\begin{ocamlindent}\ocamlinlinecode{deserialize} создаёт \hyperref[page-FPauth-core-module-FPauth+u+core-module-Router-module-Make-argument-1-M-type-t]{\ocamlinlinecode{\ocamlinlinecode{t}}}. Возвращает: \ocamlinlinecode{Ok t} если десериализация была успешна или \ocamlinlinecode{Error string} если произошла ошибка.\end{ocamlindent}%
\medbreak
\label{page-FPauth-core-module-FPauth+u+core-module-Router-module-Make-argument-1-M-val-identificate}\ocamlcodefragment{\ocamltag{keyword}{val} identificate : 
  \hyperref[xref-unresolved]{\ocamlinlinecode{Dream}}.\allowbreak{}request \ocamltag{arrow}{$\rightarrow$}
  ( \hyperref[page-FPauth-core-module-FPauth+u+core-module-Router-module-Make-argument-1-M-type-t]{\ocamlinlinecode{t}},\allowbreak{} \hyperref[xref-unresolved]{\ocamlinlinecode{Base}}.\allowbreak{}Error.\allowbreak{}t ) \hyperref[xref-unresolved]{\ocamlinlinecode{Base}}.\allowbreak{}Result.\allowbreak{}t \hyperref[xref-unresolved]{\ocamlinlinecode{Dream}}.\allowbreak{}promise}\begin{ocamlindent}\ocamlinlinecode{identificate} определяет, какая именно сущность аутентифицируется. Находит репрезентацию сущности или возвращает ошибку.\end{ocamlindent}%
\medbreak
\label{page-FPauth-core-module-FPauth+u+core-module-Router-module-Make-argument-1-M-val-applicable+u+strats}\ocamlcodefragment{\ocamltag{keyword}{val} applicable\_\allowbreak{}strats : \hyperref[page-FPauth-core-module-FPauth+u+core-module-Router-module-Make-argument-1-M-type-t]{\ocamlinlinecode{t}} \ocamltag{arrow}{$\rightarrow$} \hyperref[xref-unresolved]{\ocamlinlinecode{Base}}.\allowbreak{}string \hyperref[xref-unresolved]{\ocamlinlinecode{Base}}.\allowbreak{}list}\begin{ocamlindent}\ocamlinlinecode{applicable\_\allowbreak{}strats} возвращает список стратегий, которые могут быть применены ко всей \ocamlinlinecode{MODEL} или к определённой сущности \hyperref[page-FPauth-core-module-FPauth+u+core-module-Router-module-Make-argument-1-M-type-t]{\ocamlinlinecode{\ocamlinlinecode{t}}}. Строки должны совпадать с \ocamlinlinecode{STRATEGY}.name.\end{ocamlindent}%
\medbreak
\end{ocamlindent}%
\ocamlcodefragment{\ocamltag{keyword}{end}}\\
\label{page-FPauth-core-module-FPauth+u+core-module-Router-module-Make-argument-2-A}\ocamlcodefragment{\ocamltag{keyword}{module} \hyperref[page-FPauth-core-module-FPauth+u+core-module-Router-module-Make-argument-2-A]{\ocamlinlinecode{A}}}\ocamlcodefragment{ : \ocamltag{keyword}{sig}}\begin{ocamlindent}\label{page-FPauth-core-module-FPauth+u+core-module-Router-module-Make-argument-2-A-type-entity}\ocamlcodefragment{\ocamltag{keyword}{type} entity = \hyperref[page-FPauth-core-module-FPauth+u+core-module-Router-module-Make-argument-1-M-type-t]{\ocamlinlinecode{M.\allowbreak{}t}}}\begin{ocamlindent}тип \ocamlinlinecode{entity} - тип аутентифицируемой сущности, совпадающий с \ocamlinlinecode{MODEL}.t.\end{ocamlindent}%
\medbreak
\label{page-FPauth-core-module-FPauth+u+core-module-Router-module-Make-argument-2-A-type-strategy}\ocamlcodefragment{\ocamltag{keyword}{type} strategy = (\ocamltag{keyword}{module} \hyperref[page-FPauth-core-module-FPauth+u+core-module-Auth+u+sign-module-type-STRATEGY]{\ocamlinlinecode{Auth\_\allowbreak{}sign.\allowbreak{}STRATEGY}} \ocamltag{keyword}{with} \ocamltag{keyword}{type} \hyperref[page-FPauth-core-module-FPauth+u+core-module-Auth+u+sign-module-type-STRATEGY-type-entity]{\ocamlinlinecode{entity}} = \hyperref[page-FPauth-core-module-FPauth+u+core-module-Router-module-Make-argument-2-A-type-entity]{\ocamlinlinecode{entity}})}\begin{ocamlindent}\ocamlinlinecode{strategy} - модуль первого класса стратегии для \hyperref[page-FPauth-core-module-FPauth+u+core-module-Router-module-Make-argument-2-A-type-entity]{\ocamlinlinecode{\ocamlinlinecode{entity}}}.\end{ocamlindent}%
\medbreak
\label{page-FPauth-core-module-FPauth+u+core-module-Router-module-Make-argument-2-A-val-authenticate}\ocamlcodefragment{\ocamltag{keyword}{val} authenticate : 
  \hyperref[page-FPauth-core-module-FPauth+u+core-module-Router-module-Make-argument-2-A-type-strategy]{\ocamlinlinecode{strategy}} \hyperref[xref-unresolved]{\ocamlinlinecode{Base}}.\allowbreak{}list \ocamltag{arrow}{$\rightarrow$}
  \hyperref[xref-unresolved]{\ocamlinlinecode{Dream}}.\allowbreak{}request \ocamltag{arrow}{$\rightarrow$}
  \hyperref[page-FPauth-core-module-FPauth+u+core-module-Static-module-AuthResult-type-t]{\ocamlinlinecode{Static.\allowbreak{}AuthResult.\allowbreak{}t}} \hyperref[xref-unresolved]{\ocamlinlinecode{Dream}}.\allowbreak{}promise}\begin{ocamlindent}\ocamlinlinecode{authenticate} запускается множество стратегий для запроса и определяет, была ли аутентификация в целом успешной или нет.\end{ocamlindent}%
\medbreak
\label{page-FPauth-core-module-FPauth+u+core-module-Router-module-Make-argument-2-A-val-logout}\ocamlcodefragment{\ocamltag{keyword}{val} logout : \hyperref[xref-unresolved]{\ocamlinlinecode{Dream}}.\allowbreak{}request \ocamltag{arrow}{$\rightarrow$} \hyperref[xref-unresolved]{\ocamlinlinecode{Base}}.\allowbreak{}unit \hyperref[xref-unresolved]{\ocamlinlinecode{Lwt}}.\allowbreak{}t}\begin{ocamlindent}\ocamlinlinecode{logout} сбрасывает сессию, что приводит к сбросу статуса аутентификации. В связи с особенностями работы \ocamlinlinecode{field} текущий пользователь будет сброшен только в следующем запросе.\end{ocamlindent}%
\medbreak
\end{ocamlindent}%
\ocamlcodefragment{\ocamltag{keyword}{end}}\\
\subsubsection{Сигнатура\label{signature}}%
\label{page-FPauth-core-module-FPauth+u+core-module-Router-module-Make-type-entity}\ocamlcodefragment{\ocamltag{keyword}{type} entity = \hyperref[page-FPauth-core-module-FPauth+u+core-module-Router-module-Make-argument-1-M-type-t]{\ocamlinlinecode{M.\allowbreak{}t}}}\begin{ocamlindent}тип \ocamlinlinecode{entity} - тип аутентифицируемой сущности, совпадающий с \ocamlinlinecode{MODEL}.t.\end{ocamlindent}%
\medbreak
\label{page-FPauth-core-module-FPauth+u+core-module-Router-module-Make-type-strategy}\ocamlcodefragment{\ocamltag{keyword}{type} strategy = (\ocamltag{keyword}{module} \hyperref[page-FPauth-core-module-FPauth+u+core-module-Auth+u+sign-module-type-STRATEGY]{\ocamlinlinecode{Auth\_\allowbreak{}sign.\allowbreak{}STRATEGY}} \ocamltag{keyword}{with} \ocamltag{keyword}{type} \hyperref[page-FPauth-core-module-FPauth+u+core-module-Auth+u+sign-module-type-STRATEGY-type-entity]{\ocamlinlinecode{entity}} = \hyperref[page-FPauth-core-module-FPauth+u+core-module-Router-module-Make-type-entity]{\ocamlinlinecode{entity}})}\begin{ocamlindent}\ocamlinlinecode{strategy} - модуль первого класса стратегии для \hyperref[page-FPauth-core-module-FPauth+u+core-module-Router-module-Make-type-entity]{\ocamlinlinecode{\ocamlinlinecode{entity}}}.\end{ocamlindent}%
\medbreak
\label{page-FPauth-core-module-FPauth+u+core-module-Router-module-Make-val-login+u+handler}\ocamlcodefragment{\ocamltag{keyword}{val} login\_\allowbreak{}handler : 
  \hyperref[page-FPauth-core-module-FPauth+u+core-module-Router-module-Make-type-strategy]{\ocamlinlinecode{strategy}} \hyperref[xref-unresolved]{\ocamlinlinecode{Base}}.\allowbreak{}list \ocamltag{arrow}{$\rightarrow$}
  (\ocamltag{keyword}{module} \hyperref[page-FPauth-core-module-FPauth+u+core-module-Auth+u+sign-module-type-RESPONSES]{\ocamlinlinecode{Auth\_\allowbreak{}sign.\allowbreak{}RESPONSES}}) \ocamltag{arrow}{$\rightarrow$}
  \hyperref[xref-unresolved]{\ocamlinlinecode{Dream}}.\allowbreak{}request \ocamltag{arrow}{$\rightarrow$}
  \hyperref[xref-unresolved]{\ocamlinlinecode{Dream}}.\allowbreak{}response \hyperref[xref-unresolved]{\ocamlinlinecode{Lwt}}.\allowbreak{}t}\begin{ocamlindent}\ocamlinlinecode{login\_\allowbreak{}handler} получается список стратегий и шаблоны ответов, запускает аутентификацию и обрабатывает её результаты.\end{ocamlindent}%
\medbreak
\label{page-FPauth-core-module-FPauth+u+core-module-Router-module-Make-val-logout+u+handler}\ocamlcodefragment{\ocamltag{keyword}{val} logout\_\allowbreak{}handler : 
  (\ocamltag{keyword}{module} \hyperref[page-FPauth-core-module-FPauth+u+core-module-Auth+u+sign-module-type-RESPONSES]{\ocamlinlinecode{Auth\_\allowbreak{}sign.\allowbreak{}RESPONSES}}) \ocamltag{arrow}{$\rightarrow$}
  \hyperref[xref-unresolved]{\ocamlinlinecode{Dream}}.\allowbreak{}request \ocamltag{arrow}{$\rightarrow$}
  \hyperref[xref-unresolved]{\ocamlinlinecode{Dream}}.\allowbreak{}response \hyperref[xref-unresolved]{\ocamlinlinecode{Lwt}}.\allowbreak{}t}\begin{ocamlindent}\ocamlinlinecode{logout\_\allowbreak{}handler} сбрасывает аутентификацию для текущего пользователя.\end{ocamlindent}%
\medbreak
\label{page-FPauth-core-module-FPauth+u+core-module-Router-module-Make-val-call}\ocamlcodefragment{\ocamltag{keyword}{val} call : 
  ?root:\hyperref[xref-unresolved]{\ocamlinlinecode{Base}}.\allowbreak{}string \ocamltag{arrow}{$\rightarrow$}
  responses:(\ocamltag{keyword}{module} \hyperref[page-FPauth-core-module-FPauth+u+core-module-Auth+u+sign-module-type-RESPONSES]{\ocamlinlinecode{Auth\_\allowbreak{}sign.\allowbreak{}RESPONSES}}) \ocamltag{arrow}{$\rightarrow$}
  extractor:\hyperref[page-FPauth-core-module-FPauth+u+core-module-Static-module-Params-type-extractor]{\ocamlinlinecode{Static.\allowbreak{}Params.\allowbreak{}extractor}} \ocamltag{arrow}{$\rightarrow$}
  \hyperref[page-FPauth-core-module-FPauth+u+core-module-Router-module-Make-type-strategy]{\ocamlinlinecode{strategy}} \hyperref[xref-unresolved]{\ocamlinlinecode{Base}}.\allowbreak{}list \ocamltag{arrow}{$\rightarrow$}
  \hyperref[xref-unresolved]{\ocamlinlinecode{Dream}}.\allowbreak{}route}\begin{ocamlindent}\ocamlinlinecode{call ?root \textasciitilde{}responses \textasciitilde{}extractor strat\_\allowbreak{}list} создаёт маршруты для аутентификации, которые добавляются в \ocamlinlinecode{Dream.\allowbreak{}router}.Содержит следующие базовые маршруты:\begin{itemize}\item{"/auth" является стартовой точкой для аутентификации. Передаёт \ocamlinlinecode{strategies} в \hyperref[page-FPauth-core-module-FPauth+u+core-module-Auth+u+sign-module-type-AUTHENTICATOR-val-authenticate]{\ocamlinlinecode{\ocamlinlinecode{Auth\_\allowbreak{}sign.\allowbreak{}AUTHENTICATOR.\allowbreak{}authenticate}}}.}%
\item{"/logout" выаолняет сброс аутентификации с помощью \ocamlinlinecode{Authenticator}.logout и отвечает с использованием шаблона \hyperref[page-FPauth-core-module-FPauth+u+core-module-Auth+u+sign-module-type-RESPONSES-val-logout]{\ocamlinlinecode{\ocamlinlinecode{Auth\_\allowbreak{}sign.\allowbreak{}RESPONSES.\allowbreak{}logout}}}.}\end{itemize}%
\ocamlinlinecode{extractor} определяет способ извлечения параметров из запросов для всех запросов, связанных с аутентификацией, в том числе поступающих по маршрутам \ocamlinlinecode{STRATEGY}.routes. Подробнее в \hyperref[page-FPauth-core-module-FPauth+u+core-module-Static-module-Params-type-extractor]{\ocamlinlinecode{\ocamlinlinecode{Static.\allowbreak{}Params.\allowbreak{}extractor}}}.\ocamlinlinecode{responses} определяет, какие ответы отправлять дял запросов, поступающих по базовым маршрутам.\ocamlinlinecode{?root} определяет корневой путь для всех маршрутов, связанных с аутентификацией. По умолчанию "/".\end{ocamlindent}%
\medbreak


