\subsection{Модуль \ocamlinlinecode{Make\_\allowbreak{}Auth.\allowbreak{}Router}}\label{page-FPauth-core-module-FPauth+u+core-module-Make+u+Auth-module-Router}%
\ocamlinlinecode{Router} создаёт маршруты, необходимые для аутентификации. Содержит несколько базовых handlers и объединяет их с маршрутами из стратегий \hyperref[page-FPauth-core-module-FPauth+u+core-module-Auth+u+sign-module-type-STRATEGY-val-routes]{\ocamlinlinecode{\ocamlinlinecode{Auth\_\allowbreak{}sign.\allowbreak{}STRATEGY.\allowbreak{}routes}}}.

\label{page-FPauth-core-module-FPauth+u+core-module-Make+u+Auth-module-Router-type-entity}\ocamlcodefragment{\ocamltag{keyword}{type} entity = \hyperref[page-FPauth-core-module-FPauth+u+core-module-Make+u+Auth-argument-1-M-type-t]{\ocamlinlinecode{M.\allowbreak{}t}}}\begin{ocamlindent}тип \ocamlinlinecode{entity} - тип аутентифицируемой сущности, совпадающий с \ocamlinlinecode{MODEL}.t.\end{ocamlindent}%
\medbreak
\label{page-FPauth-core-module-FPauth+u+core-module-Make+u+Auth-module-Router-type-strategy}\ocamlcodefragment{\ocamltag{keyword}{type} strategy = (\ocamltag{keyword}{module} \hyperref[page-FPauth-core-module-FPauth+u+core-module-Auth+u+sign-module-type-STRATEGY]{\ocamlinlinecode{Auth\_\allowbreak{}sign.\allowbreak{}STRATEGY}} \ocamltag{keyword}{with} \ocamltag{keyword}{type} \hyperref[page-FPauth-core-module-FPauth+u+core-module-Auth+u+sign-module-type-STRATEGY-type-entity]{\ocamlinlinecode{entity}} = \hyperref[page-FPauth-core-module-FPauth+u+core-module-Make+u+Auth-module-Router-type-entity]{\ocamlinlinecode{entity}})}\begin{ocamlindent}\ocamlinlinecode{strategy} - модуль первого класса стратегии для \hyperref[page-FPauth-core-module-FPauth+u+core-module-Make+u+Auth-module-Router-type-entity]{\ocamlinlinecode{\ocamlinlinecode{entity}}}.\end{ocamlindent}%
\medbreak
\label{page-FPauth-core-module-FPauth+u+core-module-Make+u+Auth-module-Router-val-login+u+handler}\ocamlcodefragment{\ocamltag{keyword}{val} login\_\allowbreak{}handler : 
  \hyperref[page-FPauth-core-module-FPauth+u+core-module-Make+u+Auth-module-Router-type-strategy]{\ocamlinlinecode{strategy}} \hyperref[xref-unresolved]{\ocamlinlinecode{Base}}.\allowbreak{}list \ocamltag{arrow}{$\rightarrow$}
  (\ocamltag{keyword}{module} \hyperref[page-FPauth-core-module-FPauth+u+core-module-Auth+u+sign-module-type-RESPONSES]{\ocamlinlinecode{Auth\_\allowbreak{}sign.\allowbreak{}RESPONSES}}) \ocamltag{arrow}{$\rightarrow$}
  \hyperref[xref-unresolved]{\ocamlinlinecode{Dream}}.\allowbreak{}request \ocamltag{arrow}{$\rightarrow$}
  \hyperref[xref-unresolved]{\ocamlinlinecode{Dream}}.\allowbreak{}response \hyperref[xref-unresolved]{\ocamlinlinecode{Lwt}}.\allowbreak{}t}\begin{ocamlindent}\ocamlinlinecode{login\_\allowbreak{}handler} получается список стратегий и шаблоны ответов, запускает аутентификацию и обрабатывает её результаты.\end{ocamlindent}%
\medbreak
\label{page-FPauth-core-module-FPauth+u+core-module-Make+u+Auth-module-Router-val-logout+u+handler}\ocamlcodefragment{\ocamltag{keyword}{val} logout\_\allowbreak{}handler : 
  (\ocamltag{keyword}{module} \hyperref[page-FPauth-core-module-FPauth+u+core-module-Auth+u+sign-module-type-RESPONSES]{\ocamlinlinecode{Auth\_\allowbreak{}sign.\allowbreak{}RESPONSES}}) \ocamltag{arrow}{$\rightarrow$}
  \hyperref[xref-unresolved]{\ocamlinlinecode{Dream}}.\allowbreak{}request \ocamltag{arrow}{$\rightarrow$}
  \hyperref[xref-unresolved]{\ocamlinlinecode{Dream}}.\allowbreak{}response \hyperref[xref-unresolved]{\ocamlinlinecode{Lwt}}.\allowbreak{}t}\begin{ocamlindent}\ocamlinlinecode{logout\_\allowbreak{}handler} сбрасывает аутентификацию для текущего пользователя.\end{ocamlindent}%
\medbreak
\label{page-FPauth-core-module-FPauth+u+core-module-Make+u+Auth-module-Router-val-call}\ocamlcodefragment{\ocamltag{keyword}{val} call : 
  ?root:\hyperref[xref-unresolved]{\ocamlinlinecode{Base}}.\allowbreak{}string \ocamltag{arrow}{$\rightarrow$}
  responses:(\ocamltag{keyword}{module} \hyperref[page-FPauth-core-module-FPauth+u+core-module-Auth+u+sign-module-type-RESPONSES]{\ocamlinlinecode{Auth\_\allowbreak{}sign.\allowbreak{}RESPONSES}}) \ocamltag{arrow}{$\rightarrow$}
  extractor:\hyperref[page-FPauth-core-module-FPauth+u+core-module-Static-module-Params-type-extractor]{\ocamlinlinecode{Static.\allowbreak{}Params.\allowbreak{}extractor}} \ocamltag{arrow}{$\rightarrow$}
  \hyperref[page-FPauth-core-module-FPauth+u+core-module-Make+u+Auth-module-Router-type-strategy]{\ocamlinlinecode{strategy}} \hyperref[xref-unresolved]{\ocamlinlinecode{Base}}.\allowbreak{}list \ocamltag{arrow}{$\rightarrow$}
  \hyperref[xref-unresolved]{\ocamlinlinecode{Dream}}.\allowbreak{}route}\begin{ocamlindent}\ocamlinlinecode{call ?root \textasciitilde{}responses \textasciitilde{}extractor strat\_\allowbreak{}list} создаёт маршруты для аутентификации, которые добавляются в \ocamlinlinecode{Dream.\allowbreak{}router}.Содержит следующие базовые маршруты:\begin{itemize}\item{"/auth" является стартовой точкой для аутентификации. Передаёт \ocamlinlinecode{strategies} в \hyperref[page-FPauth-core-module-FPauth+u+core-module-Auth+u+sign-module-type-AUTHENTICATOR-val-authenticate]{\ocamlinlinecode{\ocamlinlinecode{Auth\_\allowbreak{}sign.\allowbreak{}AUTHENTICATOR.\allowbreak{}authenticate}}}.}%
\item{"/logout" выаолняет сброс аутентификации с помощью \hyperref[page-FPauth-core-module-FPauth+u+core-module-Make+u+Auth-module-Authenticator-val-logout]{\ocamlinlinecode{\ocamlinlinecode{Authenticator.\allowbreak{}logout}}}.}\end{itemize}%
\ocamlinlinecode{extractor} определяет способ извлечения параметров из запросов для всех запросов, связанных с аутентификацией, в том числе поступающих по маршрутам \ocamlinlinecode{STRATEGY}.routes. Подробнее в \hyperref[page-FPauth-core-module-FPauth+u+core-module-Static-module-Params-type-extractor]{\ocamlinlinecode{\ocamlinlinecode{Static.\allowbreak{}Params.\allowbreak{}extractor}}}.\ocamlinlinecode{responses} определяет, какие ответы отправлять дял запросов, поступающих по базовым маршрутам.\ocamlinlinecode{?root} определяет корневой путь для всех маршрутов, связанных с аутентификацией. По умолчанию "/".\end{ocamlindent}%
\medbreak


