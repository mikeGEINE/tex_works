\subsection{Модуль \ocamlinlinecode{FPauth\_\allowbreak{}core.\allowbreak{}Static}}\label{page-FPauth-core-module-FPauth+u+core-module-Static}%
\ocamlinlinecode{Static} - модуль, содержащий в себе определения статичных типов, которые не зависят от \hyperref[page-FPauth-core-module-FPauth+u+core-module-Auth+u+sign-module-type-MODEL]{\ocamlinlinecode{\ocamlinlinecode{Auth\_\allowbreak{}sign.\allowbreak{}MODEL}}}.

\ocamlinlinecode{Static} - содуль, содержащий все возможности библиотеки, которые не зависят от \ocamlinlinecode{FPauth}.Auth\_sign.MODEL

\label{page-FPauth-core-module-FPauth+u+core-module-Static-module-StratResult}\ocamlcodefragment{\ocamltag{keyword}{module} \hyperref[page-FPauth-core-module-FPauth+u+core-module-Static-module-StratResult]{\ocamlinlinecode{StratResult}}}\ocamlcodefragment{ : \ocamltag{keyword}{sig}}\begin{ocamlindent}\label{page-FPauth-core-module-FPauth+u+core-module-Static-module-StratResult-type-t}\ocamlcodefragment{\ocamltag{keyword}{type} 'a t = }\\
\begin{ocamltabular}{p{0.400\textwidth}p{0.500\textwidth}}\ocamlcodefragment{| \ocamltag{constructor}{Authenticated} \ocamltag{keyword}{of} \ocamltag{type-var}{'a}}\label{page-FPauth-core-module-FPauth+u+core-module-Static-module-StratResult-type-t.Authenticated}& Сущность была успешно аутентифицирована. Также может быть использована внутри стратегий с функцией bind аналогично \ocamlinlinecode{Ok 'a} результату. Когда возвращается в \ocamlinlinecode{FPauth}.Auth\_sign.AUTHENTICATOR, завершает процесс аутентификации.\\
\ocamlcodefragment{| \ocamltag{constructor}{Rescue} \ocamltag{keyword}{of} \hyperref[xref-unresolved]{\ocamlinlinecode{Base}}.\allowbreak{}Error.\allowbreak{}t}\label{page-FPauth-core-module-FPauth+u+core-module-Static-module-StratResult-type-t.Rescue}& Аутентификация должна быть немедленно остановлена с ошибкой.\\
\ocamlcodefragment{| \ocamltag{constructor}{Redirect} \ocamltag{keyword}{of} \hyperref[xref-unresolved]{\ocamlinlinecode{Dream}}.\allowbreak{}response \hyperref[xref-unresolved]{\ocamlinlinecode{Lwt}}.\allowbreak{}t}\label{page-FPauth-core-module-FPauth+u+core-module-Static-module-StratResult-type-t.Redirect}& Пользователь должен быть перенаправлен в соответствии с \ocamlinlinecode{response}. \ocamlinlinecode{response promise} создаётся с помощью \ocamlinlinecode{Dream.\allowbreak{}redirect}.\\
\ocamlcodefragment{| \ocamltag{constructor}{Next}}\label{page-FPauth-core-module-FPauth+u+core-module-Static-module-StratResult-type-t.Next}& Следующая стратгеия из списка в \ocamlinlinecode{FPauth}.Auth\_sign.AUTHENTICATOR должна быть исполнена.\\
\end{ocamltabular}%
\\
\begin{ocamlindent}\ocamlinlinecode{'a t} определяет результаты стратегий.\end{ocamlindent}%
\medbreak
\label{page-FPauth-core-module-FPauth+u+core-module-Static-module-StratResult-val-bind}\ocamlcodefragment{\ocamltag{keyword}{val} bind : \ocamltag{type-var}{'a} \hyperref[page-FPauth-core-module-FPauth+u+core-module-Static-module-StratResult-type-t]{\ocamlinlinecode{t}} \ocamltag{arrow}{$\rightarrow$} ( \ocamltag{type-var}{'a} \ocamltag{arrow}{$\rightarrow$} \ocamltag{type-var}{'b} \hyperref[page-FPauth-core-module-FPauth+u+core-module-Static-module-StratResult-type-t]{\ocamlinlinecode{t}} ) \ocamltag{arrow}{$\rightarrow$} \ocamltag{type-var}{'b} \hyperref[page-FPauth-core-module-FPauth+u+core-module-Static-module-StratResult-type-t]{\ocamlinlinecode{t}}}\begin{ocamlindent}\ocamlinlinecode{bind r f} возвращает \ocamlinlinecode{f r} если \ocamlinlinecode{r} является \hyperref[page-FPauth-core-module-FPauth+u+core-module-Static-module-StratResult-type-t.Authenticated]{\ocamlinlinecode{\ocamlinlinecode{Authenticated}}} или \ocamlinlinecode{r} в иных случаях.\end{ocamlindent}%
\medbreak
\label{page-FPauth-core-module-FPauth+u+core-module-Static-module-StratResult-module-Infix}\ocamlcodefragment{\ocamltag{keyword}{module} \hyperref[page-FPauth-core-module-FPauth+u+core-module-Static-module-StratResult-module-Infix]{\ocamlinlinecode{Infix}}}\ocamlcodefragment{ : \ocamltag{keyword}{sig}}\begin{ocamlindent}\label{page-FPauth-core-module-FPauth+u+core-module-Static-module-StratResult-module-Infix-val-(>>==)}\ocamlcodefragment{\ocamltag{keyword}{val} (>>==) : \ocamltag{type-var}{'a} \hyperref[page-FPauth-core-module-FPauth+u+core-module-Static-module-StratResult-type-t]{\ocamlinlinecode{t}} \ocamltag{arrow}{$\rightarrow$} ( \ocamltag{type-var}{'a} \ocamltag{arrow}{$\rightarrow$} \ocamltag{type-var}{'b} \hyperref[page-FPauth-core-module-FPauth+u+core-module-Static-module-StratResult-type-t]{\ocamlinlinecode{t}} ) \ocamltag{arrow}{$\rightarrow$} \ocamltag{type-var}{'b} \hyperref[page-FPauth-core-module-FPauth+u+core-module-Static-module-StratResult-type-t]{\ocamlinlinecode{t}}}\begin{ocamlindent}Инфиксный оператор для \ocamlinlinecode{FPauth}.Static.StratResult.bind\end{ocamlindent}%
\medbreak
\end{ocamlindent}%
\ocamlcodefragment{\ocamltag{keyword}{end}}\begin{ocamlindent}Модуль с операторами в инфиксной форме для \hyperref[page-FPauth-core-module-FPauth+u+core-module-Static-module-StratResult]{\ocamlinlinecode{\ocamlinlinecode{StratResult}}}\end{ocamlindent}%
\medbreak
\end{ocamlindent}%
\ocamlcodefragment{\ocamltag{keyword}{end}}\begin{ocamlindent}\ocamlinlinecode{StratResult} определяет результат стратегий, а также задаёт вспомогательные функции.\end{ocamlindent}%
\medbreak
\label{page-FPauth-core-module-FPauth+u+core-module-Static-module-AuthResult}\ocamlcodefragment{\ocamltag{keyword}{module} \hyperref[page-FPauth-core-module-FPauth+u+core-module-Static-module-AuthResult]{\ocamlinlinecode{AuthResult}}}\ocamlcodefragment{ : \ocamltag{keyword}{sig}}\begin{ocamlindent}\label{page-FPauth-core-module-FPauth+u+core-module-Static-module-AuthResult-type-t}\ocamlcodefragment{\ocamltag{keyword}{type} t = }\\
\begin{ocamltabular}{p{0.400\textwidth}p{0.500\textwidth}}\ocamlcodefragment{| \ocamltag{constructor}{Authenticated}}\label{page-FPauth-core-module-FPauth+u+core-module-Static-module-AuthResult-type-t.Authenticated}& Сущность была успешно аутентифицирована.\\
\ocamlcodefragment{| \ocamltag{constructor}{Rescue}}\label{page-FPauth-core-module-FPauth+u+core-module-Static-module-AuthResult-type-t.Rescue}& Аутентификация завершилась с ошибкой.\\
\ocamlcodefragment{| \ocamltag{constructor}{Redirect} \ocamltag{keyword}{of} \hyperref[xref-unresolved]{\ocamlinlinecode{Dream}}.\allowbreak{}response \hyperref[xref-unresolved]{\ocamlinlinecode{Lwt}}.\allowbreak{}t}\label{page-FPauth-core-module-FPauth+u+core-module-Static-module-AuthResult-type-t.Redirect}& Пользователь должен быть перенаправлен в соответствии с \ocamlinlinecode{response}. \ocamlinlinecode{response promise} создаётся с помощью \ocamlinlinecode{Dream.\allowbreak{}redirect}.\\
\end{ocamltabular}%
\\
\end{ocamlindent}%
\ocamlcodefragment{\ocamltag{keyword}{end}}\begin{ocamlindent}\ocamlinlinecode{AuthResult} - результат всего процесса аутентификации. Похож на \hyperref[page-FPauth-core-module-FPauth+u+core-module-Static-module-StratResult]{\ocamlinlinecode{\ocamlinlinecode{StratResult}}}, но не содержит некоторые типы, которые имеют смысл только для стратегий. \ocamlinlinecode{Authenticated} и \ocamlinlinecode{Rescue} не содержат в себе данных, так как они сохраняются в \ocamlinlinecode{Dream.\allowbreak{}field} к концу аутентификации.\end{ocamlindent}%
\medbreak
\label{page-FPauth-core-module-FPauth+u+core-module-Static-module-Params}\ocamlcodefragment{\ocamltag{keyword}{module} \hyperref[page-FPauth-core-module-FPauth+u+core-module-Static-module-Params]{\ocamlinlinecode{Params}}}\ocamlcodefragment{ : \ocamltag{keyword}{sig}}\begin{ocamlindent}\label{page-FPauth-core-module-FPauth+u+core-module-Static-module-Params-type-t}\ocamlcodefragment{\ocamltag{keyword}{type} t}\\
\label{page-FPauth-core-module-FPauth+u+core-module-Static-module-Params-val-params}\ocamlcodefragment{\ocamltag{keyword}{val} params : \hyperref[xref-unresolved]{\ocamlinlinecode{Dream}}.\allowbreak{}request \ocamltag{arrow}{$\rightarrow$} \hyperref[page-FPauth-core-module-FPauth+u+core-module-Static-module-Params-type-t]{\ocamlinlinecode{t}} option}\begin{ocamlindent}\ocamlinlinecode{params request} возвращает \hyperref[page-FPauth-core-module-FPauth+u+core-module-Static-module-Params-type-t]{\ocamlinlinecode{\ocamlinlinecode{t}}} middleware.\end{ocamlindent}%
\medbreak
\label{page-FPauth-core-module-FPauth+u+core-module-Static-module-Params-type-extractor}\ocamlcodefragment{\ocamltag{keyword}{type} extractor = \hyperref[xref-unresolved]{\ocamlinlinecode{Dream}}.\allowbreak{}request \ocamltag{arrow}{$\rightarrow$} \hyperref[page-FPauth-core-module-FPauth+u+core-module-Static-module-Params-type-t]{\ocamlinlinecode{t}} \hyperref[xref-unresolved]{\ocamlinlinecode{Lwt}}.\allowbreak{}t}\begin{ocamlindent}\ocamlinlinecode{extractor} - тип функции, которая извлекает параметры из запросов.\end{ocamlindent}%
\medbreak
\label{page-FPauth-core-module-FPauth+u+core-module-Static-module-Params-val-get+u+param}\ocamlcodefragment{\ocamltag{keyword}{val} get\_\allowbreak{}param : string \ocamltag{arrow}{$\rightarrow$} \hyperref[page-FPauth-core-module-FPauth+u+core-module-Static-module-Params-type-t]{\ocamlinlinecode{t}} \ocamltag{arrow}{$\rightarrow$} string option}\begin{ocamlindent}\ocamlinlinecode{get\_\allowbreak{}param key params} ищет заданный ключ \ocamlinlinecode{key} в \ocamlinlinecode{params} и возвращает \ocamlinlinecode{Some str}, если параметр был найден, или \ocamlinlinecode{None} в ином случае.\end{ocamlindent}%
\medbreak
\label{page-FPauth-core-module-FPauth+u+core-module-Static-module-Params-val-get+u+param+u+exn}\ocamlcodefragment{\ocamltag{keyword}{val} get\_\allowbreak{}param\_\allowbreak{}exn : string \ocamltag{arrow}{$\rightarrow$} \hyperref[page-FPauth-core-module-FPauth+u+core-module-Static-module-Params-type-t]{\ocamlinlinecode{t}} \ocamltag{arrow}{$\rightarrow$} string}\begin{ocamlindent}\ocamlinlinecode{get\_\allowbreak{}param\_\allowbreak{}exn key params} совпадает с \hyperref[page-FPauth-core-module-FPauth+u+core-module-Static-module-Params-val-get+u+param]{\ocamlinlinecode{\ocamlinlinecode{get\_\allowbreak{}param}}}, но возвращает исключение в случае, если \ocamlinlinecode{key} отсутствует.\end{ocamlindent}%
\medbreak
\label{page-FPauth-core-module-FPauth+u+core-module-Static-module-Params-val-get+u+param+u+req}\ocamlcodefragment{\ocamltag{keyword}{val} get\_\allowbreak{}param\_\allowbreak{}req : string \ocamltag{arrow}{$\rightarrow$} \hyperref[xref-unresolved]{\ocamlinlinecode{Dream}}.\allowbreak{}request \ocamltag{arrow}{$\rightarrow$} string option}\begin{ocamlindent}\ocamlinlinecode{get\_\allowbreak{}param\_\allowbreak{}req key request} является сокращением \ocamlinlinecode{params request >{}>= get\_\allowbreak{}param key}.\end{ocamlindent}%
\medbreak
\label{page-FPauth-core-module-FPauth+u+core-module-Static-module-Params-val-extract+u+query}\ocamlcodefragment{\ocamltag{keyword}{val} extract\_\allowbreak{}query : \hyperref[page-FPauth-core-module-FPauth+u+core-module-Static-module-Params-type-extractor]{\ocamlinlinecode{extractor}}}\begin{ocamlindent}\ocamlinlinecode{extract\_\allowbreak{}query request} извлекает все query-параметры запроса и возвращает их в виде \hyperref[page-FPauth-core-module-FPauth+u+core-module-Static-module-Params-type-t]{\ocamlinlinecode{\ocamlinlinecode{t}}}.\end{ocamlindent}%
\medbreak
\label{page-FPauth-core-module-FPauth+u+core-module-Static-module-Params-val-extract+u+json}\ocamlcodefragment{\ocamltag{keyword}{val} extract\_\allowbreak{}json : \hyperref[page-FPauth-core-module-FPauth+u+core-module-Static-module-Params-type-extractor]{\ocamlinlinecode{extractor}}}\begin{ocamlindent}\ocamlinlinecode{extract\_\allowbreak{}json request} извлекает все пары ключей-значений из запроса в формате JSON. \bold{Content-Type} запроса должен быть \ocamlinlinecode{application/json}.\end{ocamlindent}%
\medbreak
\label{page-FPauth-core-module-FPauth+u+core-module-Static-module-Params-val-extract+u+form}\ocamlcodefragment{\ocamltag{keyword}{val} extract\_\allowbreak{}form : ?csrf:bool \ocamltag{arrow}{$\rightarrow$} \hyperref[page-FPauth-core-module-FPauth+u+core-module-Static-module-Params-type-extractor]{\ocamlinlinecode{extractor}}}\begin{ocamlindent}\ocamlinlinecode{extract\_\allowbreak{}form request} извлекает параметры из форм, отправленных с \ocamlinlinecode{Dream.\allowbreak{}csrf\_\allowbreak{}tag}. \bold{Content-Type} запроса должен быть \ocamlinlinecode{application/x-www-form-\\urlencoded}.\end{ocamlindent}%
\medbreak
\label{page-FPauth-core-module-FPauth+u+core-module-Static-module-Params-val-of+u+assoc}\ocamlcodefragment{\ocamltag{keyword}{val} of\_\allowbreak{}assoc : (string * string) list \ocamltag{arrow}{$\rightarrow$} \hyperref[page-FPauth-core-module-FPauth+u+core-module-Static-module-Params-type-t]{\ocamlinlinecode{t}}}\begin{ocamlindent}\ocamlinlinecode{of\_\allowbreak{}assoc lst} создаёт \hyperref[page-FPauth-core-module-FPauth+u+core-module-Static-module-Params-type-t]{\ocamlinlinecode{\ocamlinlinecode{t}}}.\end{ocamlindent}%
\medbreak
\label{page-FPauth-core-module-FPauth+u+core-module-Static-module-Params-val-set+u+params}\ocamlcodefragment{\ocamltag{keyword}{val} set\_\allowbreak{}params : 
  extractor:\hyperref[page-FPauth-core-module-FPauth+u+core-module-Static-module-Params-type-extractor]{\ocamlinlinecode{extractor}} \ocamltag{arrow}{$\rightarrow$}
  \hyperref[xref-unresolved]{\ocamlinlinecode{Dream}}.\allowbreak{}handler \ocamltag{arrow}{$\rightarrow$}
  \hyperref[xref-unresolved]{\ocamlinlinecode{Dream}}.\allowbreak{}request \ocamltag{arrow}{$\rightarrow$}
  \hyperref[xref-unresolved]{\ocamlinlinecode{Dream}}.\allowbreak{}response \hyperref[xref-unresolved]{\ocamlinlinecode{Dream}}.\allowbreak{}promise}\begin{ocamlindent}\ocamlinlinecode{ser\_\allowbreak{}params \textasciitilde{}extractor} - middleware, которое устанавливает параметры для запроса, извлекая их с помощью \ocamlinlinecode{\textasciitilde{}extractor}.\end{ocamlindent}%
\medbreak
\end{ocamlindent}%
\ocamlcodefragment{\ocamltag{keyword}{end}}\begin{ocamlindent}\ocamlinlinecode{Params} хранит параметры запроса, все или только требуемые для аутентификации.\end{ocamlindent}%
\medbreak


