\subsection{Модуль \ocamlinlinecode{FPauth\_\allowbreak{}core.\allowbreak{}Make\_\allowbreak{}Auth}}\label{page-FPauth-core-module-FPauth+u+core-module-Make+u+Auth}%
\ocamlinlinecode{Make\_\allowbreak{}Auth} создаёт модуль аутентификации на основе \hyperref[page-FPauth-core-module-FPauth+u+core-module-Auth+u+sign-module-type-MODEL]{\ocamlinlinecode{\ocamlinlinecode{Auth\_\allowbreak{}sign.\allowbreak{}MODEL}}}. Предоставляет локальные переменные (\ocamlinlinecode{field}), \ocamlinlinecode{middleware}, аутентификатор, запускающий стратегии аутентификации, а также маршрутизатор, добавляющий в приложение маршруты аутентификации.

\subsubsection{Параметры\label{parameters}}%
\label{page-FPauth-core-module-FPauth+u+core-module-Make+u+Auth-argument-1-M}\ocamlcodefragment{\ocamltag{keyword}{module} \hyperref[page-FPauth-core-module-FPauth+u+core-module-Make+u+Auth-argument-1-M]{\ocamlinlinecode{M}}}\ocamlcodefragment{ : \ocamltag{keyword}{sig}}\begin{ocamlindent}\label{page-FPauth-core-module-FPauth+u+core-module-Make+u+Auth-argument-1-M-type-t}\ocamlcodefragment{\ocamltag{keyword}{type} t}\begin{ocamlindent}Некоторое представление сущности, которая будет аутентифицирована.\end{ocamlindent}%
\medbreak
\label{page-FPauth-core-module-FPauth+u+core-module-Make+u+Auth-argument-1-M-val-serialize}\ocamlcodefragment{\ocamltag{keyword}{val} serialize : \hyperref[page-FPauth-core-module-FPauth+u+core-module-Make+u+Auth-argument-1-M-type-t]{\ocamlinlinecode{t}} \ocamltag{arrow}{$\rightarrow$} \hyperref[xref-unresolved]{\ocamlinlinecode{Base}}.\allowbreak{}string}\begin{ocamlindent}\ocamlinlinecode{serialize ent} создаёт \ocamlinlinecode{string} из \hyperref[page-FPauth-core-module-FPauth+u+core-module-Make+u+Auth-argument-1-M-type-t]{\ocamlinlinecode{\ocamlinlinecode{t}}}.\end{ocamlindent}%
\medbreak
\label{page-FPauth-core-module-FPauth+u+core-module-Make+u+Auth-argument-1-M-val-deserialize}\ocamlcodefragment{\ocamltag{keyword}{val} deserialize : \hyperref[xref-unresolved]{\ocamlinlinecode{Base}}.\allowbreak{}string \ocamltag{arrow}{$\rightarrow$} ( \hyperref[page-FPauth-core-module-FPauth+u+core-module-Make+u+Auth-argument-1-M-type-t]{\ocamlinlinecode{t}},\allowbreak{} \hyperref[xref-unresolved]{\ocamlinlinecode{Base}}.\allowbreak{}Error.\allowbreak{}t ) \hyperref[xref-unresolved]{\ocamlinlinecode{Base}}.\allowbreak{}Result.\allowbreak{}t}\begin{ocamlindent}\ocamlinlinecode{deserialize} создаёт \hyperref[page-FPauth-core-module-FPauth+u+core-module-Make+u+Auth-argument-1-M-type-t]{\ocamlinlinecode{\ocamlinlinecode{t}}}. Возвращает: \ocamlinlinecode{Ok t} если десериализация была успешна или \ocamlinlinecode{Error string} если произошла ошибка.\end{ocamlindent}%
\medbreak
\label{page-FPauth-core-module-FPauth+u+core-module-Make+u+Auth-argument-1-M-val-identificate}\ocamlcodefragment{\ocamltag{keyword}{val} identificate : 
  \hyperref[xref-unresolved]{\ocamlinlinecode{Dream}}.\allowbreak{}request \ocamltag{arrow}{$\rightarrow$}
  ( \hyperref[page-FPauth-core-module-FPauth+u+core-module-Make+u+Auth-argument-1-M-type-t]{\ocamlinlinecode{t}},\allowbreak{} \hyperref[xref-unresolved]{\ocamlinlinecode{Base}}.\allowbreak{}Error.\allowbreak{}t ) \hyperref[xref-unresolved]{\ocamlinlinecode{Base}}.\allowbreak{}Result.\allowbreak{}t \hyperref[xref-unresolved]{\ocamlinlinecode{Dream}}.\allowbreak{}promise}\begin{ocamlindent}\ocamlinlinecode{identificate} определяет, какая именно сущность аутентифицируется. Находит репрезентацию сущности или возвращает ошибку.\end{ocamlindent}%
\medbreak
\label{page-FPauth-core-module-FPauth+u+core-module-Make+u+Auth-argument-1-M-val-applicable+u+strats}\ocamlcodefragment{\ocamltag{keyword}{val} applicable\_\allowbreak{}strats : \hyperref[page-FPauth-core-module-FPauth+u+core-module-Make+u+Auth-argument-1-M-type-t]{\ocamlinlinecode{t}} \ocamltag{arrow}{$\rightarrow$} \hyperref[xref-unresolved]{\ocamlinlinecode{Base}}.\allowbreak{}string \hyperref[xref-unresolved]{\ocamlinlinecode{Base}}.\allowbreak{}list}\begin{ocamlindent}\ocamlinlinecode{applicable\_\allowbreak{}strats} возвращает список стратегий, которые могут быть применены ко всей \ocamlinlinecode{MODEL} или к определённой сущности \hyperref[page-FPauth-core-module-FPauth+u+core-module-Make+u+Auth-argument-1-M-type-t]{\ocamlinlinecode{\ocamlinlinecode{t}}}. Строки должны совпадать с \ocamlinlinecode{STRATEGY}.name.\end{ocamlindent}%
\medbreak
\end{ocamlindent}%
\ocamlcodefragment{\ocamltag{keyword}{end}}\\
\subsubsection{Сигнатура\label{signature}}%
\label{page-FPauth-core-module-FPauth+u+core-module-Make+u+Auth-module-Variables}\ocamlcodefragment{\ocamltag{keyword}{module} \hyperref[page-FPauth-core-module-FPauth+u+core-module-Make+u+Auth-module-Variables]{\ocamlinlinecode{Variables}}}\ocamlcodefragment{ : \hyperref[page-FPauth-core-module-FPauth+u+core-module-Auth+u+sign-module-type-VARIABLES]{\ocamlinlinecode{Auth\_\allowbreak{}sign.\allowbreak{}VARIABLES}} \ocamltag{keyword}{with} \ocamltag{keyword}{type} \hyperref[page-FPauth-core-module-FPauth+u+core-module-Auth+u+sign-module-type-VARIABLES-type-entity]{\ocamlinlinecode{entity}} = \hyperref[page-FPauth-core-module-FPauth+u+core-module-Make+u+Auth-argument-1-M-type-t]{\ocamlinlinecode{M.\allowbreak{}t}}}\begin{ocamlindent}\ocamlinlinecode{Variables} содержит типы, функции и \ocamlinlinecode{fields}, основанные на \hyperref[page-FPauth-core-module-FPauth+u+core-module-Auth+u+sign-module-type-MODEL]{\ocamlinlinecode{\ocamlinlinecode{Auth\_\allowbreak{}sign.\allowbreak{}MODEL}}}.\end{ocamlindent}%
\medbreak
\label{page-FPauth-core-module-FPauth+u+core-module-Make+u+Auth-module-Session+u+manager}\ocamlcodefragment{\ocamltag{keyword}{module} \hyperref[page-FPauth-core-module-FPauth+u+core-module-Make+u+Auth-module-Session+u+manager]{\ocamlinlinecode{Session\_\allowbreak{}manager}}}\ocamlcodefragment{ : \hyperref[page-FPauth-core-module-FPauth+u+core-module-Auth+u+sign-module-type-SESSIONMANAGER]{\ocamlinlinecode{Auth\_\allowbreak{}sign.\allowbreak{}SESSIONMANAGER}} \ocamltag{keyword}{with} \ocamltag{keyword}{type} \hyperref[page-FPauth-core-module-FPauth+u+core-module-Auth+u+sign-module-type-SESSIONMANAGER-type-entity]{\ocamlinlinecode{entity}} = \hyperref[page-FPauth-core-module-FPauth+u+core-module-Make+u+Auth-argument-1-M-type-t]{\ocamlinlinecode{M.\allowbreak{}t}}}\begin{ocamlindent}\ocamlinlinecode{SessionManager} - модуль, который устанавливает \ocamlinlinecode{fields} из сессии для каждого запроса через \hyperref[page-FPauth-core-module-FPauth+u+core-module-Auth+u+sign-module-type-SESSIONMANAGER-val-auth+u+setup]{\ocamlinlinecode{\ocamlinlinecode{Auth\_\allowbreak{}sign.\allowbreak{}SESSIONMANAGER.\allowbreak{}auth\_\allowbreak{}setup}}} middleware.\end{ocamlindent}%
\medbreak
\label{page-FPauth-core-module-FPauth+u+core-module-Make+u+Auth-module-Authenticator}\ocamlcodefragment{\ocamltag{keyword}{module} \hyperref[page-FPauth-core-module-FPauth+u+core-module-Make+u+Auth-module-Authenticator]{\ocamlinlinecode{Authenticator}}}\ocamlcodefragment{ : \hyperref[page-FPauth-core-module-FPauth+u+core-module-Auth+u+sign-module-type-AUTHENTICATOR]{\ocamlinlinecode{Auth\_\allowbreak{}sign.\allowbreak{}AUTHENTICATOR}} \ocamltag{keyword}{with} \ocamltag{keyword}{type} \hyperref[page-FPauth-core-module-FPauth+u+core-module-Auth+u+sign-module-type-AUTHENTICATOR-type-entity]{\ocamlinlinecode{entity}} = \hyperref[page-FPauth-core-module-FPauth+u+core-module-Make+u+Auth-argument-1-M-type-t]{\ocamlinlinecode{M.\allowbreak{}t}}}\begin{ocamlindent}\ocamlinlinecode{Authenticator} содержит функции для исполнения списка \hyperref[page-FPauth-core-module-FPauth+u+core-module-Auth+u+sign-module-type-STRATEGY]{\ocamlinlinecode{\ocamlinlinecode{Auth\_\allowbreak{}sign.\allowbreak{}STRATEGY}}} и сброса аутентификации.\end{ocamlindent}%
\medbreak
\label{page-FPauth-core-module-FPauth+u+core-module-Make+u+Auth-module-Router}\ocamlcodefragment{\ocamltag{keyword}{module} \hyperref[page-FPauth-core-module-FPauth+u+core-module-Make+u+Auth-module-Router]{\ocamlinlinecode{Router}}}\ocamlcodefragment{ : \hyperref[page-FPauth-core-module-FPauth+u+core-module-Auth+u+sign-module-type-ROUTER]{\ocamlinlinecode{Auth\_\allowbreak{}sign.\allowbreak{}ROUTER}} \ocamltag{keyword}{with} \ocamltag{keyword}{type} \hyperref[page-FPauth-core-module-FPauth+u+core-module-Auth+u+sign-module-type-ROUTER-type-entity]{\ocamlinlinecode{entity}} = \hyperref[page-FPauth-core-module-FPauth+u+core-module-Make+u+Auth-argument-1-M-type-t]{\ocamlinlinecode{M.\allowbreak{}t}}}\begin{ocamlindent}\ocamlinlinecode{Router} создаёт маршруты, необходимые для аутентификации. Содержит несколько базовых handlers и объединяет их с маршрутами из стратегий \hyperref[page-FPauth-core-module-FPauth+u+core-module-Auth+u+sign-module-type-STRATEGY-val-routes]{\ocamlinlinecode{\ocamlinlinecode{Auth\_\allowbreak{}sign.\allowbreak{}STRATEGY.\allowbreak{}routes}}}.\end{ocamlindent}%
\medbreak

\subsection{Модуль \ocamlinlinecode{Make\_\allowbreak{}Auth.\allowbreak{}Variables}}\label{page-FPauth-core-module-FPauth+u+core-module-Make+u+Auth-module-Variables}%
\ocamlinlinecode{Variables} содержит типы, функции и \ocamlinlinecode{fields}, основанные на \hyperref[page-FPauth-core-module-FPauth+u+core-module-Auth+u+sign-module-type-MODEL]{\ocamlinlinecode{\ocamlinlinecode{Auth\_\allowbreak{}sign.\allowbreak{}MODEL}}}.

\label{page-FPauth-core-module-FPauth+u+core-module-Make+u+Auth-module-Variables-type-entity}\ocamlcodefragment{\ocamltag{keyword}{type} entity = \hyperref[page-FPauth-core-module-FPauth+u+core-module-Make+u+Auth-argument-1-M-type-t]{\ocamlinlinecode{M.\allowbreak{}t}}}\begin{ocamlindent}тип \ocamlinlinecode{entity} - тип аутентифицируемой сущности, совпадающий с \ocamlinlinecode{MODEL}.t.\end{ocamlindent}%
\medbreak
\label{page-FPauth-core-module-FPauth+u+core-module-Make+u+Auth-module-Variables-val-authenticated}\ocamlcodefragment{\ocamltag{keyword}{val} authenticated : \hyperref[xref-unresolved]{\ocamlinlinecode{Base}}.\allowbreak{}bool \hyperref[xref-unresolved]{\ocamlinlinecode{Dream}}.\allowbreak{}field}\begin{ocamlindent}\ocamlinlinecode{authenticated} - переменная, действительная в рамках одного запроса, отражает, была ли пройдена аутентификация ранее. Устанавливается в \ocamlinlinecode{SESSIONMANAGER}.auth\_setup.\end{ocamlindent}%
\medbreak
\label{page-FPauth-core-module-FPauth+u+core-module-Make+u+Auth-module-Variables-val-current+u+user}\ocamlcodefragment{\ocamltag{keyword}{val} current\_\allowbreak{}user : \hyperref[page-FPauth-core-module-FPauth+u+core-module-Make+u+Auth-module-Variables-type-entity]{\ocamlinlinecode{entity}} \hyperref[xref-unresolved]{\ocamlinlinecode{Dream}}.\allowbreak{}field}\begin{ocamlindent}\ocamlinlinecode{current\_\allowbreak{}user} - переменная, действительная в рамках одного запроса, содержит аутентифицированную сущность (если ранее была пройдена аутентификация). Устанавливается в \ocamlinlinecode{SESSIONMANAGER}.auth\_setup\end{ocamlindent}%
\medbreak
\label{page-FPauth-core-module-FPauth+u+core-module-Make+u+Auth-module-Variables-val-auth+u+error}\ocamlcodefragment{\ocamltag{keyword}{val} auth\_\allowbreak{}error : \hyperref[xref-unresolved]{\ocamlinlinecode{Base}}.\allowbreak{}Error.\allowbreak{}t \hyperref[xref-unresolved]{\ocamlinlinecode{Dream}}.\allowbreak{}field}\begin{ocamlindent}\ocamlinlinecode{auth\_\allowbreak{}error} - field-переменная с ошибкой, которая могла произойти на любом этапе аутентификации. Устанавливается в \ocamlinlinecode{AUTHENTICATOR}.authenticate.\end{ocamlindent}%
\medbreak
\label{page-FPauth-core-module-FPauth+u+core-module-Make+u+Auth-module-Variables-val-update+u+current+u+user}\ocamlcodefragment{\ocamltag{keyword}{val} update\_\allowbreak{}current\_\allowbreak{}user : \hyperref[page-FPauth-core-module-FPauth+u+core-module-Make+u+Auth-module-Variables-type-entity]{\ocamlinlinecode{entity}} \ocamltag{arrow}{$\rightarrow$} \hyperref[xref-unresolved]{\ocamlinlinecode{Dream}}.\allowbreak{}request \ocamltag{arrow}{$\rightarrow$} \hyperref[xref-unresolved]{\ocamlinlinecode{Base}}.\allowbreak{}unit \hyperref[xref-unresolved]{\ocamlinlinecode{Dream}}.\allowbreak{}promise}\begin{ocamlindent}\ocamlinlinecode{update\_\allowbreak{}current\_\allowbreak{}user user request} обновляет \hyperref[page-FPauth-core-module-FPauth+u+core-module-Make+u+Auth-module-Variables-val-current+u+user]{\ocamlinlinecode{\ocamlinlinecode{current\_\allowbreak{}user}}} и сессию. Необходимо использовать в том случе, если были внесены изменения, влияющие на сериализацию.\end{ocamlindent}%
\medbreak



\subsection{Модуль \ocamlinlinecode{Make\_\allowbreak{}Auth.\allowbreak{}Session\_\allowbreak{}manager}}\label{page-FPauth-core-module-FPauth+u+core-module-Make+u+Auth-module-Session+u+manager}%
\ocamlinlinecode{SessionManager} - модуль, который устанавливает \ocamlinlinecode{fields} из сессии для каждого запроса через \hyperref[page-FPauth-core-module-FPauth+u+core-module-Auth+u+sign-module-type-SESSIONMANAGER-val-auth+u+setup]{\ocamlinlinecode{\ocamlinlinecode{Auth\_\allowbreak{}sign.\allowbreak{}SESSIONMANAGER.\allowbreak{}auth\_\allowbreak{}setup}}} middleware.

\label{page-FPauth-core-module-FPauth+u+core-module-Make+u+Auth-module-Session+u+manager-type-entity}\ocamlcodefragment{\ocamltag{keyword}{type} entity = \hyperref[page-FPauth-core-module-FPauth+u+core-module-Make+u+Auth-argument-1-M-type-t]{\ocamlinlinecode{M.\allowbreak{}t}}}\begin{ocamlindent}тип \ocamlinlinecode{entity} - тип аутентифицируемой сущности, совпадающий с \ocamlinlinecode{MODEL}.t\end{ocamlindent}%
\medbreak
\label{page-FPauth-core-module-FPauth+u+core-module-Make+u+Auth-module-Session+u+manager-val-auth+u+setup}\ocamlcodefragment{\ocamltag{keyword}{val} auth\_\allowbreak{}setup : \hyperref[xref-unresolved]{\ocamlinlinecode{Dream}}.\allowbreak{}middleware}\begin{ocamlindent}\ocamlinlinecode{auth\_\allowbreak{}setup} - middleware, которое контролирует сессию, устанавливает переменные-\ocamlinlinecode{field} и всопмогательные функции для последующих handlers.\ocamlinlinecode{auth\_\allowbreak{}setup} пробует извлечь строку из сессии с ключом \ocamlinlinecode{auth} и определить статус аутентификации. Если поле \ocamlinlinecode{auth} отсутствует, то аутентификация не была пройдена. Если \ocamlinlinecode{auth} имеется, то по строке проверяется и изменяется статус аутентификации:\begin{itemize}\item{Если \ocamlinlinecode{auth} содержит пустую строку, то ситуация считается ошибочной;}%
\item{Если \hyperref[page-FPauth-core-module-FPauth+u+core-module-Make+u+Auth-argument-1-M-val-deserialize]{\ocamlinlinecode{\ocamlinlinecode{M.\allowbreak{}deserialize}}} вернула \ocamlinlinecode{Error Error.\allowbreak{}t}, то аутентификация не пройдена и ситуация считается ошибочной;}%
\item{Если \hyperref[page-FPauth-core-module-FPauth+u+core-module-Make+u+Auth-argument-1-M-val-deserialize]{\ocamlinlinecode{\ocamlinlinecode{M.\allowbreak{}deserialize}}} вернула \ocamlinlinecode{Ok M.\allowbreak{}t}, то аутентификация считается успешной и устанавливаются \ocamlinlinecode{VARIABLES}.current\_user и \ocamlinlinecode{VARIABLES}.authenticated. Если с сессией что-то не так, то сессия становится недействительной, ошибка записывается и статус 401 отправляется. Если сессия в порядке, то запрос поступает в следующий обработчик.}\end{itemize}%
\end{ocamlindent}%
\medbreak



\input{FPauth-core/FPauth_core.Make_Auth.Authenticator.tex}
\subsection{Модуль \ocamlinlinecode{Make\_\allowbreak{}Auth.\allowbreak{}Router}}\label{page-FPauth-core-module-FPauth+u+core-module-Make+u+Auth-module-Router}%
\ocamlinlinecode{Router} создаёт маршруты, необходимые для аутентификации. Содержит несколько базовых handlers и объединяет их с маршрутами из стратегий \hyperref[page-FPauth-core-module-FPauth+u+core-module-Auth+u+sign-module-type-STRATEGY-val-routes]{\ocamlinlinecode{\ocamlinlinecode{Auth\_\allowbreak{}sign.\allowbreak{}STRATEGY.\allowbreak{}routes}}}.

\label{page-FPauth-core-module-FPauth+u+core-module-Make+u+Auth-module-Router-type-entity}\ocamlcodefragment{\ocamltag{keyword}{type} entity = \hyperref[page-FPauth-core-module-FPauth+u+core-module-Make+u+Auth-argument-1-M-type-t]{\ocamlinlinecode{M.\allowbreak{}t}}}\begin{ocamlindent}тип \ocamlinlinecode{entity} - тип аутентифицируемой сущности, совпадающий с \ocamlinlinecode{MODEL}.t.\end{ocamlindent}%
\medbreak
\label{page-FPauth-core-module-FPauth+u+core-module-Make+u+Auth-module-Router-type-strategy}\ocamlcodefragment{\ocamltag{keyword}{type} strategy = (\ocamltag{keyword}{module} \hyperref[page-FPauth-core-module-FPauth+u+core-module-Auth+u+sign-module-type-STRATEGY]{\ocamlinlinecode{Auth\_\allowbreak{}sign.\allowbreak{}STRATEGY}} \ocamltag{keyword}{with} \ocamltag{keyword}{type} \hyperref[page-FPauth-core-module-FPauth+u+core-module-Auth+u+sign-module-type-STRATEGY-type-entity]{\ocamlinlinecode{entity}} = \hyperref[page-FPauth-core-module-FPauth+u+core-module-Make+u+Auth-module-Router-type-entity]{\ocamlinlinecode{entity}})}\begin{ocamlindent}\ocamlinlinecode{strategy} - модуль первого класса стратегии для \hyperref[page-FPauth-core-module-FPauth+u+core-module-Make+u+Auth-module-Router-type-entity]{\ocamlinlinecode{\ocamlinlinecode{entity}}}.\end{ocamlindent}%
\medbreak
\label{page-FPauth-core-module-FPauth+u+core-module-Make+u+Auth-module-Router-val-login+u+handler}\ocamlcodefragment{\ocamltag{keyword}{val} login\_\allowbreak{}handler : 
  \hyperref[page-FPauth-core-module-FPauth+u+core-module-Make+u+Auth-module-Router-type-strategy]{\ocamlinlinecode{strategy}} \hyperref[xref-unresolved]{\ocamlinlinecode{Base}}.\allowbreak{}list \ocamltag{arrow}{$\rightarrow$}
  (\ocamltag{keyword}{module} \hyperref[page-FPauth-core-module-FPauth+u+core-module-Auth+u+sign-module-type-RESPONSES]{\ocamlinlinecode{Auth\_\allowbreak{}sign.\allowbreak{}RESPONSES}}) \ocamltag{arrow}{$\rightarrow$}
  \hyperref[xref-unresolved]{\ocamlinlinecode{Dream}}.\allowbreak{}request \ocamltag{arrow}{$\rightarrow$}
  \hyperref[xref-unresolved]{\ocamlinlinecode{Dream}}.\allowbreak{}response \hyperref[xref-unresolved]{\ocamlinlinecode{Lwt}}.\allowbreak{}t}\begin{ocamlindent}\ocamlinlinecode{login\_\allowbreak{}handler} получается список стратегий и шаблоны ответов, запускает аутентификацию и обрабатывает её результаты.\end{ocamlindent}%
\medbreak
\label{page-FPauth-core-module-FPauth+u+core-module-Make+u+Auth-module-Router-val-logout+u+handler}\ocamlcodefragment{\ocamltag{keyword}{val} logout\_\allowbreak{}handler : 
  (\ocamltag{keyword}{module} \hyperref[page-FPauth-core-module-FPauth+u+core-module-Auth+u+sign-module-type-RESPONSES]{\ocamlinlinecode{Auth\_\allowbreak{}sign.\allowbreak{}RESPONSES}}) \ocamltag{arrow}{$\rightarrow$}
  \hyperref[xref-unresolved]{\ocamlinlinecode{Dream}}.\allowbreak{}request \ocamltag{arrow}{$\rightarrow$}
  \hyperref[xref-unresolved]{\ocamlinlinecode{Dream}}.\allowbreak{}response \hyperref[xref-unresolved]{\ocamlinlinecode{Lwt}}.\allowbreak{}t}\begin{ocamlindent}\ocamlinlinecode{logout\_\allowbreak{}handler} сбрасывает аутентификацию для текущего пользователя.\end{ocamlindent}%
\medbreak
\label{page-FPauth-core-module-FPauth+u+core-module-Make+u+Auth-module-Router-val-call}\ocamlcodefragment{\ocamltag{keyword}{val} call : 
  ?root:\hyperref[xref-unresolved]{\ocamlinlinecode{Base}}.\allowbreak{}string \ocamltag{arrow}{$\rightarrow$}
  responses:(\ocamltag{keyword}{module} \hyperref[page-FPauth-core-module-FPauth+u+core-module-Auth+u+sign-module-type-RESPONSES]{\ocamlinlinecode{Auth\_\allowbreak{}sign.\allowbreak{}RESPONSES}}) \ocamltag{arrow}{$\rightarrow$}
  extractor:\hyperref[page-FPauth-core-module-FPauth+u+core-module-Static-module-Params-type-extractor]{\ocamlinlinecode{Static.\allowbreak{}Params.\allowbreak{}extractor}} \ocamltag{arrow}{$\rightarrow$}
  \hyperref[page-FPauth-core-module-FPauth+u+core-module-Make+u+Auth-module-Router-type-strategy]{\ocamlinlinecode{strategy}} \hyperref[xref-unresolved]{\ocamlinlinecode{Base}}.\allowbreak{}list \ocamltag{arrow}{$\rightarrow$}
  \hyperref[xref-unresolved]{\ocamlinlinecode{Dream}}.\allowbreak{}route}\begin{ocamlindent}\ocamlinlinecode{call ?root \textasciitilde{}responses \textasciitilde{}extractor strat\_\allowbreak{}list} создаёт маршруты для аутентификации, которые добавляются в \ocamlinlinecode{Dream.\allowbreak{}router}.Содержит следующие базовые маршруты:\begin{itemize}\item{"/auth" является стартовой точкой для аутентификации. Передаёт \ocamlinlinecode{strategies} в \hyperref[page-FPauth-core-module-FPauth+u+core-module-Auth+u+sign-module-type-AUTHENTICATOR-val-authenticate]{\ocamlinlinecode{\ocamlinlinecode{Auth\_\allowbreak{}sign.\allowbreak{}AUTHENTICATOR.\allowbreak{}authenticate}}}.}%
\item{"/logout" выаолняет сброс аутентификации с помощью \hyperref[page-FPauth-core-module-FPauth+u+core-module-Make+u+Auth-module-Authenticator-val-logout]{\ocamlinlinecode{\ocamlinlinecode{Authenticator.\allowbreak{}logout}}}.}\end{itemize}%
\ocamlinlinecode{extractor} определяет способ извлечения параметров из запросов для всех запросов, связанных с аутентификацией, в том числе поступающих по маршрутам \ocamlinlinecode{STRATEGY}.routes. Подробнее в \hyperref[page-FPauth-core-module-FPauth+u+core-module-Static-module-Params-type-extractor]{\ocamlinlinecode{\ocamlinlinecode{Static.\allowbreak{}Params.\allowbreak{}extractor}}}.\ocamlinlinecode{responses} определяет, какие ответы отправлять дял запросов, поступающих по базовым маршрутам.\ocamlinlinecode{?root} определяет корневой путь для всех маршрутов, связанных с аутентификацией. По умолчанию "/".\end{ocamlindent}%
\medbreak



