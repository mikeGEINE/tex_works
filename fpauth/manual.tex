% !TEX program = xelatex

\documentclass[a4paper, 12pt]{article}
\usepackage{bmstu-lab}
\hypersetup{
    linkcolor=black,
    urlcolor=black
}
\usepackage[english,russian]{babel}
% \usepackage{showframe}
% \usepackage{fontspec}
% \usepackage{xunicode}

\usepackage{changepage}
\usepackage{longtable}
\usepackage{listings}
\usepackage[strings]{underscore}

% \usepackage[colorlinks=true]{hyperref}
\usepackage{color}
% \usepackage{lmodern}
\usepackage[T1]{fontenc}
\usepackage{import}
% \usepackage{courier}

\newfontfamily\monofamily{Courier New}

\newcommand{\ocamlcodefragment}[1]{{\monofamily\setlength{\parindent}{0cm}\raggedright#1}}
\newcommand{\ocamlinlinecode}[1]{{\monofamily#1}}
\newcommand{\bold}[1]{{\bfseries#1}}
\newenvironment{ocamlexception}{\bfseries}{}
\newenvironment{ocamlextension}{\bfseries}{}

\newenvironment{ocamlkeyword}{\bfseries}{}

\newenvironment{ocamlconstructor}{\bfseries}{}
\newenvironment{ocamltype-var}{\itshape\monofamily}{}

\newcommand{\ocamlhighlight}{\bfseries\uline}
\newcommand{\ocamlerror}{\bfseries}
\newcommand{\ocamlwarning}{\bfseries}

\newcommand{\ocamltag}[2]{\begin{ocaml#1}#2\end{ocaml#1}}

\definecolor{lightgray}{gray}{0.97}
\definecolor{gray}{gray}{0.5}
\newcommand{\ocamlcomment}{\color{gray}\normalfont\small}
\newcommand{\ocamlstring}{\color{gray}\bfseries}
\newenvironment{ocamlindent}{\begin{adjustwidth}{2em}{0pt}}{\end{adjustwidth}}
\newenvironment{ocamltabular}[1]{\begin{tabular}{#1}}
{\end{tabular}}

\lstnewenvironment{ocamlcodeblock}{
  \lstset{
    backgroundcolor = \color{lightgray},
    basicstyle=\monofamily,
    showstringspaces=false,
    language=caml,
    escapeinside={$}{$},
    columns=fullflexible,
    stringstyle=\ocamlstring,
    commentstyle=\ocamlcomment,
    keepspaces=true,
    keywordstyle=\ocamlkeyword,
    moredelim=[is][\ocamlhighlight]{<<}{>>},
    moredelim=[s][\ocamlstring]{\{|}{|\}},
    moredelim=[s][\ocamlstring]{\{delimiter|}{|delimiter\}},
    keywords={[2]{val,initializer,nonrec}}, keywordstyle={[2]\ocamlkeyword},
    belowskip=0\baselineskip,
    upquote=true,
    breaklines=true,
    literate={'"'}{\textquotesingle "\textquotesingle}3
    {'\\"'}{\textquotesingle \textbackslash"\textquotesingle}4,
  }
  }{}
\newcommand{\inputchapter}[1]{\chapter{#1}
\input{#1}}

\newenvironment{ocamlarrow}{\bfseries}{}

\begin{document}
    \graphicspath{{images/}{images2/}} % папки с картинками
    \renewcommand{\figurename}{Рисунок}

    \addition{ПРИЛОЖЕНИЕ Б}
    \workname{ПРОГРАММНАЯ БИБЛИОТЕКА ПОДПРОГРАММ ИДЕНТИФИКАЦИИ ПОЛЬЗОВАТЕЛЕЙ}
    \worktype{Руководство системного программиста}
    \group{ИУ6-84Б}
    \author{М.А. Гейне}
    \tutor[Руководитель]{П. В. Аргентов}
    \bmstutitle
    \tolerance=1000

    \tableofcontents
    \pagebreak

    \section{FPauth-strategies - стратегии аутентификации\label{fpauth-strategies---стратегии-аутентификации}}\label{page-FPauth-strategies-leaf-page-index}%
В данном пакете содержатся 2 стратегии аутентификации: парольная и TOTP. Подробнее в разделе
\hyperref[page-FPauth-strategies-module-FPauth+u+strategies]{\ocamlinlinecode{\ocamlinlinecode{FPauth\_\allowbreak{}strategies}}}{}.



    \subimport{codes/}{FPauth.tex}
    \pagebreak

    \section{FPauth-strategies - стратегии аутентификации\label{fpauth-strategies---стратегии-аутентификации}}\label{page-FPauth-strategies-leaf-page-index}%
В данном пакете содержатся 2 стратегии аутентификации: парольная и TOTP. Подробнее в разделе
\hyperref[page-FPauth-strategies-module-FPauth+u+strategies]{\ocamlinlinecode{\ocamlinlinecode{FPauth\_\allowbreak{}strategies}}}{}.



    \subimport{codes/}{FPauth_core.tex}
    \pagebreak

    \section{FPauth-strategies - стратегии аутентификации\label{fpauth-strategies---стратегии-аутентификации}}\label{page-FPauth-strategies-leaf-page-index}%
В данном пакете содержатся 2 стратегии аутентификации: парольная и TOTP. Подробнее в разделе
\hyperref[page-FPauth-strategies-module-FPauth+u+strategies]{\ocamlinlinecode{\ocamlinlinecode{FPauth\_\allowbreak{}strategies}}}{}.



    \subimport{codes/}{FPauth_strategies.tex}
    \pagebreak

    \section{FPauth-strategies - стратегии аутентификации\label{fpauth-strategies---стратегии-аутентификации}}\label{page-FPauth-strategies-leaf-page-index}%
В данном пакете содержатся 2 стратегии аутентификации: парольная и TOTP. Подробнее в разделе
\hyperref[page-FPauth-strategies-module-FPauth+u+strategies]{\ocamlinlinecode{\ocamlinlinecode{FPauth\_\allowbreak{}strategies}}}{}.



    \subimport{codes/}{FPauth_responses.tex}
    \pagebreak

    \section{Полная версия}
    С более подробной версией руководства можно ознакомиться \href{https://mikegeine.github.io/FPauth/}{online}\footnote{\url{https://mikegeine.github.io/FPauth/}}.
    
\end{document}