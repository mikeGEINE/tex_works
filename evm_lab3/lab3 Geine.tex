% !TEX program = xelatex

\documentclass[a4paper, 14pt]{article}
\usepackage{bmstu-lab}
\hypersetup{
    linkcolor=black
}
% \usepackage{showframe}


\begin{document}
    \graphicspath{{images/}{images2/}} % папки с картинками
    \renewcommand{\figurename}{Рисунок}

    \worknumber{3}
    \variant{67}
    \workname{Проектирование устройств
    управления на основе ПЛИС}
    \discipline{Основы проектирования устройств ЭВМ}
    \group{ИУ6-64Б}
    \author{М.А.Гейне}
    \tutor[Преподаватель]{А.Ю.Попов}
    \bmstutitlelab

    \textbf{Цель работы:} закрепление на практике теоретических знаний о способах
    реализации устройств управления, исследование способов организации узлов ЭВМ,
    освоение принципов проектирования цифровых устройств на основе ПЛИС.
    \section*{Задание}
    В ходе выполнения лабораторной работы необходимо разработать устройство
    управления схемного типа, обрабатывающее входное командное слово
    $ C ={ABCDEF}$ и выдающее сигналы управления $M={M_0,…,M_{k-1}}$
    операционному блоку в соответствии с приведенной в индивидуальном задании
    логикой работы.
    
    Задание выполняется на основе выполненного домашнего задания, в рамках которого было разработано устройство управления.
    В лабораторной работе необходимо разработанное устройство подготовить для работы на ПЛИС и прошить плату.

    \section*{Вариант 67}
    \begin{table}[H]
        \caption{Варианты диаграмм и активных сигналов}
        \begin{tabular}{|c|c|c|l|l|l|l|l|}
            \hline
            \multirow{2}{*}{Вариант} & \multirow{2}{*}{Диаграмма переходов} & \multicolumn{6}{c|}{Активные сигналы M в состоянии}                                                                                                      \\ \cline{3-8} 
                                     &                                      & S1                     & \multicolumn{1}{c|}{S2} & \multicolumn{1}{c|}{S3} & \multicolumn{1}{c|}{S4} & \multicolumn{1}{c|}{S5} & \multicolumn{1}{c|}{S6} \\ \hline
            \multicolumn{1}{|l|}{67} & \multicolumn{1}{l|}{3}               & \multicolumn{1}{l|}{2} & 0                       & 1,7                     & 5,6                     & 3                       & 4                       \\ \hline
        \end{tabular}
    \end{table}

    \begin{table}[H]
        \caption{Условия переходов и наименование отладочной платы}  
        \begin{tabular}{|l|l|l|l|l|l|l|}
            \hline
            \multicolumn{1}{|c|}{\multirow{2}{*}{\textbf{Вариант}}} & \multicolumn{1}{c|}{\multirow{2}{*}{\textbf{Название отладочной платы}}} & \multicolumn{5}{c|}{\textbf{Активные сигналы C в состоянии}}                                                                                                                 \\ \cline{3-7} 
            \multicolumn{1}{|c|}{}                                  & \multicolumn{1}{c|}{}                                                    & \multicolumn{1}{c|}{\textbf{У1}} & \multicolumn{1}{c|}{\textbf{У2}} & \multicolumn{1}{c|}{\textbf{У3}} & \multicolumn{1}{c|}{\textbf{У4}} & \multicolumn{1}{c|}{\textbf{У5}} \\ \hline
            \multirow{5}{*}{67}                                     & \multirow{5}{*}{Spartan3}                                                & @                                & E\_F                             & CD                               & \_A\_C                           & @                                \\ \cline{3-7} 
                                                                    &                                                                          & \multicolumn{1}{c|}{\textbf{У6}} & \textbf{У7}                      & \textbf{У8}                      & \textbf{У9}                      & \textbf{У10}                     \\ \cline{3-7} 
                                                                    &                                                                          & D\_B                             & @                                & A\_C                              & C\_D                            & @                                \\ \cline{3-7} 
                                                                    &                                                                          & \textbf{У11}                     & \textbf{У12}                     & \textbf{У13}                     & \textbf{У14}                     & \textbf{У15}                     \\ \cline{3-7} 
                                                                    &                                                                          & A+C                              & \_B                               & @                                & DF                            & @                              \\ \hline
            \end{tabular}  
    \end{table}
    
    \begin{table}[H]
        \caption{Активные сигналы для переходов}
        \begin{tabular}{|l|l|l|l|l|l|}
        \hline
        \multicolumn{1}{|c|}{\multirow{2}{*}{\textbf{Вариант}}} & \multicolumn{5}{c|}{\textbf{Активные сигналы M в состоянии}}                                                                                                                 \\ \cline{2-6} 
        \multicolumn{1}{|c|}{}                                  & \multicolumn{1}{c|}{\textbf{У1}} & \multicolumn{1}{c|}{\textbf{У2}} & \multicolumn{1}{c|}{\textbf{У3}} & \multicolumn{1}{c|}{\textbf{У4}} & \multicolumn{1}{c|}{\textbf{У5}} \\ \hline
        \multirow{5}{*}{67}                                     &                                  &                                  &                                  &                                  &                                  \\ \cline{2-6} 
                                                                & \multicolumn{1}{c|}{\textbf{У6}} & \textbf{У7}                      & \textbf{У8}                      & \textbf{У9}                      & \textbf{У10}                     \\ \cline{2-6} 
                                                                &                                  &                                  &                                  & 5, 6                             & 5, 7                             \\ \cline{2-6} 
                                                                & \textbf{У11}                     & \textbf{У12}                     & \textbf{У13}                     & \textbf{У14}                     & \textbf{У15}                     \\ \cline{2-6} 
                                                                &                                  & 4                                &                                  &                                  &                                  \\ \hline
        \end{tabular}
    \end{table}

    \imageinsert{general transitions scheme}{0.7}{Общая диаграмма переходов}{fig:gen_trans}
    \pagebreak

    \section{Предварительные сведения об автомате}
    В домашнем задании было установлено, что заданный автомат является автоматом смешанного типа. Диаграмма переходов его состояний приведена на рисунке \ref{fig:trans_diag}. 
    В соответствии с вариантом домашнего задания была составлена диаграмма переходов состояний автомата, приведённая на рисунке \ref{fig:trans_diag}.
    Автомат имеет асинхронные входы и выходы. Код устройства приведён ниже.
    \codeinsert{'codes/state_machine.vhd'}

    \section{Схема подавления дребезга}
    Для того, чтобы было удобно контролировать переходы автомата, в качестве тактирующего сигнала clk было решено подавать не сигналы генератора, а сигналы нажатия кнопки.
    Однако физическая кнопка имеет дребезг, который необходимо устранить. Для этого решено задействовать схему подавления дребезга, разработанную в лабораторной работе №2.
    Код устройства приведён ниже.
    \codeinsert{'codes/delay_module.vhd'}

    \section{Сборка устройства}
    В целях упрощения объединения элементов был создан schematic-файл, в котором были добавлены разработанные ранее устройства, обозначены входные и выходные порты, установлены соедиения между элементами.
    Схема устройства приведена на рисунке \ref{fig:schematic}
    \imageinsert{schematic}{0.7}{Схема устройства}{fig:schematic}
    Для работы устройства необходимо назначить портам устройства корректные места подключения на плате. Так, для порта U необходимо подключить переключатели на плате, 
    к порту btn\_ck подключить кнопку, а к выходу автомата подключить светодиоды. Файл ограничений, реализующий это, приведён ниже.
    \codeinsert{'codes/main.ucf'}
    После этого был сгенерирован файл для программирования, плата была прошита. Корректность работы автомата была проверена вручную.
    
    На переключателях платы задавались входные сигналы $ C ={ABCDEF}$. При нажатии на кнопку выполнялся переход в соответствии с диаграммой.
    При этом было проверено, что входы и выходы автомата являются асинхронными: в некоторых состояниях наблюдалось, как при изменении состояния переключателей менялось состояние светодиодов. При этом кнопка на плате не нажималась.

    \section*{Выводы}
    Изучены принципы проектирования устройств управлнения с жесткой логикой.

    Изучены способы задания автоматов Мили и Мура на языке VHDL.

    Изучены устройства с синхронными и асинхронными входами и выходами.

    Изучены методы описания устройств в виде схем в среде ISE.

    Разработано устройство, демонстрирующее работу устройства управления с жесткой логикой. Входные сигналы задавались на переключателях платы, выходные наблюдались на светодиодах. Переходы состояний производились в момент нажатия на кнопку.

    Проведено программирование разработанного проекта на ПЛИС.
    
    Проведено ручное тестирование разработанного устройства на плате. По результатам тестирования было установлено, что разработанное устройство работает корректно.
\end{document}