\documentclass[a4paper, 12pt]{article}
\usepackage{cmap}
\usepackage{mathtext}
\usepackage{amsmath,amssymb}
\usepackage[T2A]{fontenc}
\usepackage[utf8]{inputenc}
\usepackage[english, russian]{babel}
\usepackage{bmstu-lab}
\usepackage{lipsum}
\usepackage{graphicx}
\usepackage{diagbox}
\usepackage{caption}
\usepackage{subcaption}

\DeclareCaptionLabelFormat{mysubfigure}{Рис.~#2}
\captionsetup[subfigure]{labelformat=mysubfigure,labelsep=colon}

\renewcommand\thesubfigure{\arabic{subfigure}}


\begin{document}
\graphicspath{{images/}{images2/}} % папки с картинками

\worknumber{2}
\variant{4}
\workname{Рисунки и таблицы в \LaTeX}
\discipline{Автоматизация процессов разработки научно-технической документации}
\group{ИУ6-64Б}
\author{М.А.Гейне}
\tutor[Преподаватель]{Т.А.Ким}
\bmstutitlelab

\hypersetup{
    linkcolor=black
}


\section{Задание 1}
\begin{tabular}{c|c|c}
    $\times$ & & $\bigcirc$ \\ \hline
    & $\times$ & $\bigcirc$ \\ \hline
    $\times$ & $\bigcirc$ & $\times$
\end{tabular}

\pagebreak

\section{Задание 2}
\begin{table}[h]
    \captionsetup{type=figure} % Судя по всему, subcaption смотрит на тип родителя при выборе типа подписи
    % В итоге разницы между subcaptionbox и subfigure нет, тип подписи всё равно определён "родителем"
    % Кроме того, в данном случае "родителем" является столбец. По этой причине нужно указывать тип в каждом столбце, но при смене типа нумерация сбивается
    % Решение проблемы: объявить тип для всей таблицы - фигура (нумерация не сбивается), при необходимости подписать таблицу - сменить перед этип тип обратно на таблицу
    \begin{tabular}{p{230pt}p{230pt}}
        \subcaptionbox{Картинка сверху}[\linewidth]{\includegraphics[height=2cm,keepaspectratio]{jake.jpg}}
        \lipsum[1]
        \begin{subfigure}[b]{\linewidth}
            \centering
            \includegraphics[height=2cm,keepaspectratio]{jake.jpg}
            \caption{Картинка снизу}
        \end{subfigure}    
        &
        \lipsum[1]
        \subcaptionbox{Картинка в месте размещения}[\linewidth]{\includegraphics[height=2cm,keepaspectratio]{jake.jpg}}
    \end{tabular}
\end{table}

\pagebreak

\section{Задание 3}

\begin{table}[h]
    \begin{center}
        \begin{tabular}{||cccccccccc||}
            \hline
            \hline
            & & & & & & & & &  \\
            \cline{2-6}
            & \multicolumn{5}{|c|}{ } & & & &  \\
            \cline{2-6}
            & & & & & & & & & \\
            & & & & & & & & &  \\
            & &  & & & & & & &  \\
            & & & & & & & & &  \\
            & & & & & & & & &  \\
            & & & & & & & & &  \\
            \cline{5-5}
            & & & & \multicolumn{1}{|c|}{ } & & & & &  \\
            \cline{5-5}
            & & & & & & & & &  \\
            \hline \hline
        \end{tabular}
        \caption{Змейка}
    \end{center}  
\end{table}

\section{Задание 4}

\begin{table}[!h]
    \begin{center}
        \begin{tabular}{|c|c|c|c|}
            \hline
            \multicolumn{2}{|c|}{\multirow{2}{*}{\backslashbox{4}{1}}}& \multicolumn{2}{|c|}{23} \\
            \cline{3-4}
            \multicolumn{2}{|c|}{} &5 &6 \\
            \hline
            \multicolumn{2}{|c|}{6} &7 &8 \\
            \hline
        \end{tabular}
        \caption{Таблица с объединением ячеек}\label{tab:ref}
    \end{center}
\end{table}
В таблице \ref{tab:ref} на странице \pageref{tab:ref} есть заголовок с диагональным разделителем. 

\end{document}
