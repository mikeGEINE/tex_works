% !TEX program = xelatex

\documentclass[a4paper, 14pt]{extarticle}
\usepackage[english, russian]{babel}
\usepackage{fontspec} %% подготавливает загрузку шрифтов Open Type, True Type и др.
\defaultfontfeatures{Ligatures={TeX},Renderer=Basic} %% свойства шрифтов по умолчанию
\setmainfont[Ligatures={TeX,Historic}]{Times New Roman} %% зада¨eт основной шрифт документа
\setsansfont{Comic Sans MS} %% задаёт шрифт без засечек
\setmonofont{Courier New}
\usepackage{indentfirst} % отменяет отсутствие абзацного отступа для первого абзаца.
\frenchspacing % Правильные отступы для кавычек, пробелов, знаков препинания.
\usepackage{bmstu-lab}
\usepackage{fancyvrb}
\usepackage{fvextra}
\usepackage{verbatim}
\usepackage{float}
\hypersetup{
    linkcolor=black
}
\usepackage{tikz}
\usetikzlibrary{shapes,shapes.multipart}
\usetikzlibrary{shapes.geometric, arrows}


\begin{document}
    \graphicspath{{images/}{images2/}} % папки с картинками

    \worknumber{5}
    \variant{4}
    \workname{Программная графика TikZ}
    \discipline{Автоматизация процессов разработки научно-технической документации}
    \group{ИУ6-64Б}
    \author{М.А.Гейне}
    \tutor[Преподаватель]{Т.А.Ким}
    \bmstutitlelab

    \textbf{Цель работы:} 
    получить навыки по использованию LaTeX как инструмента для получения векторных изображений, заданных программно.

    \section{Код задания}
    \VerbatimInput[frame=lines, breaklines=true,
    breakanywhere=true]{'Lab5 Geine Work.tex'}

    
    \section{Результаты выполнения}
    \subsection{Задание 1}
\begin{figure}[H]
    \centering
    \begin{tikzpicture}
        \draw [help lines, step = 0.5] (-1.5, 0) grid (5, 5);
        \draw[->, thick] (-1.5, 0) -- (5,0) node[right] {$x$};
        \draw[->, thick] (0, 0) -- (0,5) node[above] {$f(x)$};
        \foreach \x in {-1,...,4}
            \draw (\x, -0.1) -- (\x, 0.1) node[below] {$\x$};
        \foreach \x in {1,...,4}
            \draw (-0.1, \x) -- (0.1, \x) node[left] {$\x$};
        \draw [red, very thick, domain = -1:4.5] plot (\x, {-2/5*abs(\x)+3})
            node[right] {$f(x)= -\frac{2}{5} \left\lvert x \right\rvert +3 $};
    \end{tikzpicture}
    \caption{График функции}
\end{figure}

\subsection{Задание 2}
\tikzstyle{startstop} = [rectangle, rounded corners, minimum width=3cm, minimum height=1cm,text centered, draw=black, fill=red!30]
\tikzstyle{io} = [trapezium, trapezium left angle=70, trapezium right angle=110, minimum width=3cm, minimum height=1cm, text centered, draw=black, fill=blue!30]
\tikzstyle{process} = [rectangle, minimum width=3cm, minimum height=1cm, text centered,draw=black, fill=orange!30]
\tikzstyle{decision} = [diamond, minimum width=3cm, minimum height=1cm, text centered, draw=black, fill=green!30, aspect=2.5]
\tikzstyle{arrow} = [thick,->,>=stealth]
\usetikzlibrary{positioning}
\begin{figure}[H]
    \centering
    \begin{tikzpicture}[node distance=2cm, scale=0.3]
        \node (start) [startstop] {Начало};
        \node (in) [io, below of=start] {Загрузка A, B и n};
        \node (reset_c) [process, below of=in] {$C=0$};
        \node (check_b) [decision, below of=reset_c] {$b_0=1$?};
        \node (sum) [process, below of=check_b] {$C=C+A$};
        \node (shift) [process, below of=sum] {Сдвиг вправо $C$ и $B$};
        \node (dec) [process, below of=shift] {$n=n-1$};
        \node (check_end) [decision, below of=dec] {$n=0$?};
        \node (out) [io, below of=check_end] {Вывод $C=A*B$};
        \node (stop) [startstop, below of=out] {Конец};
        \draw[arrow] (start) -- (in);
        \draw[arrow] (in) -- (reset_c);
        \draw[arrow] (reset_c) -- (check_b);
        \path (check_b) -- (sum) coordinate[midway] (yes_place);
        \draw[arrow] (check_b) -- (yes_place) node[right] {Да} -- (sum);
        \draw[arrow] (sum) -- (shift);
        \draw[arrow] (shift) -- (dec);
        \draw[arrow] (dec) -- (check_end);
        \path (check_end) -- (out) coordinate[midway] (yes_place1);
        \draw[arrow] (check_end) -- (yes_place1) node[right] {Да} -- (out);
        \draw[arrow] (out) -- (stop);
        \path (reset_c) -- (check_b) coordinate[midway] (loop);
        \coordinate [left=2cm of check_end] (loop_coord);
        \draw[arrow] (check_end)  -- (loop_coord) node[below]{Нет} |- (loop);
        \path (sum) -- (shift) coordinate[midway] (skip);
        \coordinate [right=1cm of check_b] (skip_coord);
        \draw[arrow] (check_b) -- (skip_coord) node[above]{Нет} |- (skip);
    \end{tikzpicture}
    \caption{Схема алгоритма умножения, начиная с младших разрядов множителя}
\end{figure}
\end{document}