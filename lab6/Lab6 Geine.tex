% !TEX program = xelatex

\documentclass[a4paper, 14pt]{extarticle}
\usepackage[english, russian]{babel}
\usepackage{fontspec} %% подготавливает загрузку шрифтов Open Type, True Type и др.
\defaultfontfeatures{Ligatures={TeX},Renderer=Basic} %% свойства шрифтов по умолчанию
\setmainfont[Ligatures={TeX,Historic}]{Times New Roman} %% зада¨eт основной шрифт документа
\setsansfont{Comic Sans MS} %% задаёт шрифт без засечек
\setmonofont[Scale=0.75]{Courier New}
\usepackage{indentfirst} % отменяет отсутствие абзацного отступа для первого абзаца.
\frenchspacing % Правильные отступы для кавычек, пробелов, знаков препинания.
\usepackage{bmstu-lab}
\usepackage{fancyvrb}
\usepackage{fvextra}
\usepackage{verbatim}
\usepackage{float}
\hypersetup{
    linkcolor=black
}
\usepackage{pdfpages}


\begin{document}
    \graphicspath{{images/}{images2/}} % папки с картинками

    \worknumber{6}
    \variant{4}
    \workname{Презентации в beamer}
    \discipline{Автоматизация процессов разработки научно-технической документации}
    \group{ИУ6-64Б}
    \author{М.А.Гейне}
    \tutor[Преподаватель]{Т.А.Ким}
    \bmstutitlelab

    \textbf{Цель работы:} 
    получить навыки по использованию LaTeX как инструмента для создания строгих презентаций.

    \section{Код задания}
    \VerbatimInput[frame=lines, breaklines=true,
    breakanywhere=true]{'Lab6 Geine Work.tex'}

    
    \section{Результаты выполнения}
    \includepdf[nup=1x3, delta=5mm 5mm, scale=0.9, frame,
    pages=-]{Lab6 Geine Work.pdf}
\end{document}