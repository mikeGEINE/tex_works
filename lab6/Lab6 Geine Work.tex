\documentclass[t]{beamer}
\usetheme{Szeged}
\usecolortheme{beaver}
% \beamertemplatenavigationsymbolsempty
\usepackage{cmap}					% поиск в PDF
\usepackage{mathtext} 				% русские буквы в формулах
\usepackage[T2A]{fontenc}			% кодировка
\usepackage[utf8]{inputenc}			% кодировка исходного текста
\usepackage[english,russian]{babel}	% локализация и переносы
\usepackage{amsmath,amsfonts,amssymb,amsthm,mathtools} % AMS
\usepackage{icomma} % "Умная" запятая: $0,2$ --- число, $0, 2$ --- перечисление
\usepackage{tikz} % Работа с графикой
\usetikzlibrary{shapes,shapes.multipart}
\usetikzlibrary{shapes.geometric, arrows}
\usepackage{pgfplots}
\usepackage{pgfplotstable}
\usepackage{float}
\usepackage{graphicx}

\title{Лабораторная работа 7}
\subtitle{Документы и презентации в \LaTeX}
\date{\today}
\author{Гейне М. А.}
\tikzstyle{startstop} = [rectangle, rounded corners, minimum width=3cm, minimum height=0.7cm,text centered, draw=black, fill=red!30]
\tikzstyle{io} = [trapezium, trapezium left angle=70, trapezium right angle=110, minimum width=3cm, minimum height=0.7cm, text centered, draw=black, fill=blue!30]
\tikzstyle{process} = [rectangle, minimum width=3cm, minimum height=0.7cm, text centered,draw=black, fill=orange!30]
\tikzstyle{decision} = [diamond, minimum width=3cm, minimum height=0.7cm, text centered, draw=black, fill=green!30, aspect=3.5]
\tikzstyle{arrow} = [thick,->,>=stealth]
\usetikzlibrary{positioning}

\begin{document}

    \frame[plain]{\titlepage}

    \begin{frame}
        \frametitle{Графика в TikZ}
        \framesubtitle{Схема алгоритма умножения, начиная с младших разрядов множителя}
        \begin{figure}[H]
            \centering
            \scalebox{0.5}{
                \begin{tikzpicture}[node distance=1.2cm]
                    \node (start) [startstop] {Начало};
                    \node (in) [io, below of=start] {Загрузка A, B и n};
                    \draw[arrow] (start) -- (in);
                    \pause

                    \node(reset_c) [process, below of=in] {$C=0$};
                    \draw[arrow] (in) -- (reset_c);

                    \uncover<3->{
                        \node (check_b) [decision, below of=reset_c] {$b_0=1$?};
                        \draw[arrow] (reset_c) -- (check_b);

                        \node (sum) [process, below of=check_b] {$C=C+A$};
                        \draw[arrow] (check_b) -- (sum);

                        \node (shift) [process, below of=sum] {Сдвиг вправо $C$ и $B$};
                        \draw[arrow] (sum) -- (shift);
                        \path (sum) -- (shift) coordinate[midway] (skip);
                        \coordinate [right=1cm of check_b] (skip_coord);
                        \draw[arrow] (check_b) -- (skip_coord) |- (skip);

                        \node (dec) [process, below of=shift] {$n=n-1$};
                        \draw[arrow] (shift) -- (dec);
                        
                        \node (check_end) [decision, below of=dec] {$n=0$?};
                        \draw[arrow] (dec) -- (check_end);
                        \path (reset_c) -- (check_b) coordinate[midway] (loop);
                        \coordinate [left=2cm of check_end] (loop_coord);
                        \draw[arrow] (check_end) -- (loop_coord) |- (loop);

                    }

                    \uncover<4->{
                        \node (out) [io, below of=check_end] {Вывод $C=A*B$};
                        \draw[arrow] (check_end) -- (out);
    
                        \node (stop) [startstop, below of=out] {Конец};
                        \draw[arrow] (out) -- (stop);
                    }
                \end{tikzpicture}
            }
            \caption{Схема алгоритма умножения}
        \end{figure}
    \end{frame}
\end{document}
